% WARNING! This file is autogenerated, do not change it by hand! Change texte/nuetzliches/lernraeume_iz.dokuwiki or the Makefile instead!
\begin{tabularx}{\textwidth}{|X|p{5cm}|p{3.6cm}|p{4cm}|}
\hline Raum & Öffnungszeiten & Ausstattung & Anmerkung \\
\hline Plaza des Informatikzentrums (Erdgeschoss und 1. Stock) & normal & Tische und Stühle, Steckdosen unter Bodenabdeckungen & \\
\hline Fachgruppenraum Informatik, IZ 150 & Der Raum ist offen, falls mindestens eine Person mit Schlüssel anwesend ist / aufgeschlossen hat. In der Vorlesungszeit ist das ab 10:00 Uhr sehr wahrscheinlich. & Kaffemaschine, Kühlschrank mit Getränken,  Sofas, Tische, WLAN, Steckdosen in Massen sowie Ethernetkabel & Die Wohnzimmeratmosphäre kann vom Lernen abhalten. Ansonsten sind aber häufig Leute da, die Fragen beantworten können. \\
\hline Fachgruppenraum Wirtschaftsinformatik, IZ 159 & nach Absprache mit Mitgliedern des Fachgruppenrates Wirtschaftsinformatik & Sofas, Tische, WLAN und Steckdosen & Nähere Informationen sind bei dem Fachgruppenrat Wirtschaftsinformatik zu erfragen. \\
\hline CIP Pool, IZ G40 & normal & Rechner-Pool mit Linux-PCs, Tafel & \\
\hline Seminarraum, IZ033 & Solange nicht anders belegt & Tische, Stühle, WLAN & Der Raum wird auch für Vorlesungen, Übungen, Seminare, Lerntreffs, … genutzt. Diese Veranstaltungen haben Priorität. Ein Schlüssel für den Raum kann im Sekreteriat des IRP geliehen werden. \\
\hline Flur vor IZ033 & normal & 6-8 Tische, 20-25 Stühle, ein Kaffeeautomat & Die Plätze liegen im Galeriegeschoss, zur Plaza hin. Es gibt also vergleichsweise wenig Tageslicht. \\
\hline
\end{tabularx}
