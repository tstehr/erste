% !TEX root = ../../1-te.tex

\subsection{Lernräume}


Hier wollen wir dir eine aktuelle Übersicht über Lernräume an der TU Braunschweig geben. 
Die Liste ist im Moment nicht vollständig. 
In unserem Wiki\footnote{\verUrl{8}{https://wiki.fginfo.tu-bs.de/doku.php?id=infos:studium:lernraeume}} pflegen wir eine Liste, die wir immer dann erweitern, wenn wir einen neuen Lernraum finden. 
Wenn du im Laufe deines Studiums einen guten Ort findest, kannst du uns den Raum mitteilen, wir überprüfen das und nehmen ihn dann in die Liste auf.

Alle Gebäude stehen von \textbf{7:30 bis 19:30 Uhr} offen, wenn nicht anders in Anlage 1 der Hausordnung der TU Braunschweig\footnote{\verUrl{8}{http://www.tu-braunschweig.de/Medien-DB/gdp/tu-ho.pdf}} erwähnt.

\subsubsection*{Informatikzentrum}

\tocheck{8}{Infos zu Lernräumen aus Wiki in die *.dokuwiki-dateien kopieren! \url{https://fginfo.cs.tu-bs.de/wiki/doku.php?id=infos:studium:lernraeume}}
% WARNING! This file is autogenerated, do not change it by hand! Change texte/nuetzliches/lernraeume_iz.dokuwiki or the Makefile instead!
\begin{tabularx}{\textwidth}{|X|p{5cm}|p{3.6cm}|p{4cm}|}
\hline Raum & Öffnungszeiten & Ausstattung & Anmerkung \\
\hline Plaza des Informatikzentrums (Erdgeschoss und 1. Stock) & normal & Tische und Stühle, Steckdosen unter Bodenabdeckungen & \\
\hline Fachgruppenraum Informatik, IZ 150 & Der Raum ist offen, falls mindestens eine Person mit Schlüssel anwesend ist / aufgeschlossen hat. In der Vorlesungszeit ist das ab 10:00 Uhr sehr wahrscheinlich. & Kaffemaschine, Kühlschrank mit Getränken,  Sofas, Tische, WLAN, Steckdosen in Massen sowie Ethernetkabel & Die Wohnzimmeratmosphäre kann vom Lernen abhalten. Ansonsten sind aber häufig Leute da, die Fragen beantworten können. \\
\hline Fachgruppenraum Wirtschaftsinformatik, IZ 159 & nach Absprache mit Mitgliedern des Fachgruppenrates Wirtschaftsinformatik & Sofas, Tische, WLAN und Steckdosen & Nähere Informationen sind bei dem Fachgruppenrat Wirtschaftsinformatik zu erfragen. \\
\hline CIP Pool, IZ G40 & normal & Rechner-Pool mit Linux-PCs, Tafel & \\
\hline Seminarraum, IZ033 & Solange nicht anders belegt & Tische, Stühle, WLAN & Der Raum wird auch für Vorlesungen, Übungen, Seminare, Lerntreffs, … genutzt. Diese Veranstaltungen haben Priorität. Ein Schlüssel für den Raum kann im Sekreteriat des IRP geliehen werden. \\
\hline Flur vor IZ033 & normal & 6-8 Tische, 20-25 Stühle, ein Kaffeeautomat & Die Plätze liegen im Galeriegeschoss, zur Plaza hin. Es gibt also vergleichsweise wenig Tageslicht. \\
\hline
\end{tabularx}


\subsubsection*{Besonderheit Fachgruppenraum: Wohn- statt Lernzimmer}
Unser Fachgruppenraum IZ 149/150 taucht zwar in der Liste auf, allerdings eher um dir das \enquote{Wohnzimmer} vieler Informatikstudierenden  zu empfehlen. Wenn du  Hilfe von höheren Semestern brauchst, mal eine Runde kickern oder etwas chillen möchtest, ist der Raum sehr zu empfehlen. Außerdem finden da unsere wöchentlichen Treffen statt. Lernen kann man dort allerdings leider ziemlich vergessen! Gerade weil der Raum als sozialer Treffpunkt fungiert, kann Mensch dort gut die Pausen verbringen, insbesondere wenn der Koffeinentzug sich bemerkbar macht. Gleiches gilt, wenn du eine Frage hast oder jemanden zum Quatschen sucht. Ungestörtes Arbeiten ist hier schwieriger, weil du so gut wie nie alleine bist und die Lärmquellen zahlreich. :)

\subsubsection*{Andere Lernräume}

% WARNING! This file is autogenerated, do not change it by hand! Change texte/nuetzliches/lernraeume_andere.dokuwiki or scripts/dokuwiki_table_to_tex.sh instead!
\begin{tabularx}{\textwidth}{|X|p{5cm}|p{3.6cm}|p{4cm}|}
\hline Raum               & Öffnungszeiten                                                                                  & Ausstattung                                                             & Anmerkung                                                                                                                                                      \\
\hline Grotrian           & eigentlich normal                                                                               & Alte Tische und Stühle, vereinzelt Tafeln                               & Wenn Mitglieder der verschiedenen Fachgruppen anwesend sind, hat das Grotrian meist länger offen. Da dies oft der Fall ist, kann man hier meist lange lernen.  \\
\hline Bibliothek         & Mo-Fr: 07-24, Sa: 10-22, So: 10-17\footnote{\url{https://ub.tu-braunschweig.de/wir_ueber_uns/standorte.php}}  & Niedrige Tische und Stühle, Rechnerarbeitsplätze, Kopierer              & Man muss leise sein, daher praktisch nicht zum Lernen in der Gruppe geeignet                                                                                   \\
\hline Mensa / Cafeteria  & Mo-Do: 08-20, Fr: 08-15\footnote{\url{http://www.sw-bs.de/braunschweig/essen}}                                & Tische, Stühle, WLAN, Verpflegung inkl. Selbstbedienungs-Kaffeeautomat  & Die Plätze sind primär zum Essen gedacht, von Lernsessions zu den Stoßzeiten sollte man also im eigenen und fremden Interesse absehen.                         \\
\hline Audimax Vorräume   & normal                                                                                          & Tische, Stühle, sehr wenige schwer zu findende Steckdosen               &                                                                                                                                                                \\
\hline Forumsgebäude EG   & Mo-Fr: 06-20                                                                                    & Tische, Stühle, keine Steckdosen                                        &                                                                                                                                                                \\
\hline Studihaus (Wendenring 1 – 4) & Mo-Fr: 08-20 & barrierefrei zugänglich & Zugang nur mit Studiausweis\\
\hline
\end{tabularx}

