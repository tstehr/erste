% !TEX root = ../../1-te.tex

\subsection{Studienplan}
	\label{bach_studienplan}
	\tocheck{8}{Stimmt die Anzahl Pflichtveranstaltungen noch? Mit aktuellem Stundenplan abgleichen!}
	Wie du wahrscheinlich bereits in deinem Stundenplan festgestellt hast, musst du im ersten Semester \iftoggle{winter}{vier}{drei} Pflichtveranstaltungen hören. Doch die Bezeichnung Pflichtverantstaltung sagt bloß aus, dass du die Veranstaltung \emph{irgendwann} einmal hören musst, um deinen Bachelor abzuschließen. Die zeitliche Abfolge der Veranstaltungen darfst du aber selbst festlegen. Der Fakultäts-Musterstudienplan bietet hier eine gute Orientierungsmöglichkeit. Du musst dich aber nicht daran halten. Niemand zwingt dich eine Veranstaltung zu hören oder hält dich davon ab. Du kannst dich eigentlich in jede Vorlesung setzen, auch ohne hinterher an der Prüfung teilnehmen zu müssen -- allerdings gibt es dann auch keine Punkte dafür. Hier bieten sich zum Beispiel Module aus dem Wahlplichtbereich Informatik an, die eventuell nur alle 2 Jahre angeboten werden und über mehrere Semester gehen. Bei den (Pflicht-)Modulen der Informatik musst du jedoch beachten, dass einige Module auf anderen aufbauen. Zum Beispiel sollten Programmiergrundlagen in den ersten zwei Semestern erarbeitet werden und mit Theoretische Informatik II wirst du dich schwer tun, wenn du TheoInf I nicht gehört hast.

	Damit sich dein Studium nicht unnötig verlängert, solltest du darauf achten, in jedem Semester rund 30 Leistungspunkte zu erwerben.
