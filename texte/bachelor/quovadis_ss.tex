\subsection{Quo vadis studens?}

Wie geht das eigentlich, studieren?\\
Wie du lernst, studierst, lebst; ob du brav mitschreibst oder öfter mal ausschläfst kannst und musst du selbst entscheiden. \\
Wann du die vorgeschriebenen Lehrveranstaltungen belegst, liegt ebenfalls in deinem eigenen Ermessen, allerdings: Nachdem, bis auf vier Ausnahmen, klar festgelegt ist was du studieren musst, ergibt sich eine sinnvolle Reihenfolge, da beispielsweise fortgeschrittenes Programmieren ohne Kenntnis von Algorithmen schlicht nicht möglich ist. Nichtsdestotrotz hast du Spielraum, das Studium an deine persönliche Situation anzupassen.
Du wohnst noch zu Hause und brauchst nicht arbeiten? Prima, mach noch Technische Informatik 2 im 1. Semester. Du hast ein Kind und musst nebenbei auch noch arbeiten? Kein Problem, sprich dich mit deinem Mentor ab und mach ein Teilzeitstudium. Die konkreten Vorschriften zum Studium findest du in der Prüfungsordnung.

In kurz: Grundsätzlich musst du Veranstaltungen im Wert von 180 Credit Points (CP) erfolgreich absolvieren, davon 116–121 CP im Bereich Informatik, 35 CP in Mathematik, 14–19 CP für dein Nebenfach und 10 CP für Schlüsselqualifikationen.

Um dir einen sinnvollen Weg durchs Studium zu ermöglichen, gibt es von der Fakultät den Musterstudienplan, der versucht Überschneidungen der Veranstaltungen zu vermeiden. Es gibt aber auch noch einen Alternativstudienplan der Fachgruppe, diesen empfehlen wir dir allerdings nur, wenn du dir den geringen Mehraufwand pro Semester zutraust. Dafür wirst du es sehr genießen, während des SEPs und der Bachelorarbeit nicht so viele Vorlesungen zu haben. 
Du bist  nicht mehr in der Schule, du  hast nun Freiheiten, nutz sie
weise und studiere so, wie du es für richtig hälst. 
\\ \\
\subsection{Quo vadis? - Wo geht die Reise hin?}
	Das Leben und Lernen an der Uni ist sehr spannend. Es bieten sich viele Möglichkeiten, das Studium individuell zu gestalten, nach Interessen zu wählen und schließlich den erwünschten Abschluss zu erhalten. Ein Student genießt große Freiheiten. Aus großen Freiheiten ergibt sich aber auch eine große Verantwortung. Wie das zusammenhängt und welche Gefahren daraus resultieren, soll hier einmal kurz aufgearbeitet werden.
	
	Grundsätzlich gilt an der Uni zunächst, dich zwingt niemand irgendetwas zu machen. Vorlesungen können besucht werden, müssen aber nicht. Hausaufgaben können gemacht werden, müssen aber nicht. Prüfungen können abgelegt werden, müssen aber nicht.
	
	Dieses Konzept spiegelt eine gewisse Scheinfreiwilligkeit wieder, die es aber gar nicht ist. Der spannende Unterschied ist der folgende: "Dich zwingt niemand etwas zu tun." heißt noch lange nicht "freiwillig"! Um Prüfungen erfolgreich zu bestehen, Punkte zu sammeln und schlussendlich einen Abschluss zu bekommen, muss gelernt werden. Das ist das Hauptziel im Studium. Die spannende Frage ist daher: "Wie gehst du mit dieser neuen Situation um?"
	
	Schauen wir uns dazu einen einfachen Grundsatz an. In den Vorlesungen werden die wichtigen, theoretischen Inhalte vermittelt. In den Übungen werden Aufgaben und Herangehensweisen zu dem Stoff der Vorlesung vermittelt. Beides ist wichtiges Wissen, dass für die Prüfung am Ende des Semesters benötigt wird.
	
	Ziel muss es im Semester also sein, den Stoff zu verstehen, zu lernen und in der Klausur auf Aufgaben anwenden zu können, egal ob Veranstaltungen besucht werden oder nicht.Klar, manche Vorlesungen sind gähnend langweilig, manche Vorlesungen sind viel zu theoretisch und manchen Dozenten kann einfach nicht zugehört werden. Das sind alles Gründe, irgendwann nicht mehr in die Vorlesung zu gehen, aber dann fehlt eben ein wichtiger Teil des Lernens. "Ich kann doch ein oder zwei Bücher lesen und mir das Wissen selber aneignen." Ja, das ist richtig, das kannst du machen. Für einige mag dies tatsächlich der bessere Weg sein, aber im großen und ganzen ist dies viel mühsamer als die Vorlesung zu besuchen.
	
	Was heißt das jetzt genau? \\
	Das heißt eigentlich nur eines: Lass dich von deinen neu gewonnen Freiheiten nicht daran hindern, erfolgreich zu studieren. Besuche lieber einmal mehr die Vorlesung als das eine mal zu wenig. Gerade in den ersten Semestern ist dies wärmstens von uns empfohlen. \\
	Es gilt immer der Grundsatz $ kein Zwang \neq freiwillig $ (dt: Kein Zwang ist nicht gleich freiwillig).
	
\begin{center}
\includegraphics[totalheight=6cm]{bilder/XKCD/priorities}
\end{center}

%	Denn dass es keinen Zwang hier gibt, heißt leider nicht, dass
%:wLernen freiwillig ist und man nichts tun muss. 
%	doch das kleinere ÜKlar, manche Vorlesungen sind gähnend langweilig, manche Vorlesungen sind viel zu theoretisch und manchen Dozenten kann einfach nicht zugehört werden. Das sind alles Gründe, irgendwann nicht mehr in die Vorlesung zu gehen, aber dann fehlt eben ein wichtiger Teil des Lernens. "Ich kann doch ein oder zwei Bücher lesen und mir das Wissen selber aneignen." Ja, das ist richtig, das kannst du machen. Für einige mag dies tatsächlich der bessere Weg sein, aber im großen und ganzen ist dies viel mühsamer als die Vorlesung zu besuchen.
	
%	Was heißt das jetzt genau? \\
%	Das heißt eigentlich nur eines: Lass dich von deinen neu gewonnen Freiheiten nicht daran hindern, erfolgreich zu studieren. Besuche lieber einmal mehr die Vorlesung als das eine mal zu wenig. Gerade in den ersten Semestern ist dies wärmstens von uns empfohlen. \\
%	Es gilt immer der Grundsatz $ kein Zwang \neq freiwillig $ (dt: Kein Zwang ist nicht gleich freiwillig).
