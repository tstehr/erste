\subsection{Eure Veranstaltungen im ersten \\ Bachelor-Semester}
	Um euch einen kleinen Vorgeschmack auf die Themen zu geben, die euch im ersten Semester beschäftigen werden, gibt es hier einen Überblick:

\subsubsection{Algorithmen und Datenstrukturen}
	\textit{Prof. S\'andor Fekete}
	Diese Vorlesung vermittelt Programmiersprachenunabhängige Algorithmen und Konzepte wie Bäume, Listen oder Stacks. Wer nicht weiß, was sich hinter diesen Begriffen verbirgt, sollte auf keinen Fall die Übungen verpassen.

\subsubsection{Programmieren 1}
	\textit{Dr. Werner Struckmann}
	Programmiert wird hier fast ausschließlich in Java. Wer keine oder nur wenig Erfahrungen mit Java gemacht hat, sollte unbedingt die kleinen Übungen bearbeiten.
	In Programmieren 1 (jährlich im Wintersemester) geht es um grundlegende Konzepte der Programmierung am Beispiel von Java. Darauf aufbauend wird in Programmieren 2 (jährlich im 	Sommersemester) die Implementierung von Algorithmen und Datenstrukturen geübt. 

\subsubsection{Lineare Algebra}
	\textit{Dr. Wolfgang Marten}
	Hier geht es um Vektoren und Matrizen, sowie ein wenig Gruppentheorie. Die Übungen sind zwar nicht immer einfach, geben aber einen sehr guten Ausblick auf die Klausur.

\subsubsection{Diskrete Mathematik}
	\textit{Prof. Arnfried Kemnitz und PD JP Bode}
	Diskrete Mathematik handelt von allem, was mit ganzen Zahlen zu tun hat: Fibbonacci-Zahlen, Primzahlen, Modulorechnung, usw. Es werden die wichtigsten Mathemaischen Grundlagen vermittelt, unter anderem in Logik, Kombinatorik, Zahlentheorie und Algebra. Die kleinen Übungen sind hier eine sehr gute Vorbereitung auf die Hausaufgaben und die Klausur.

\subsubsection{Theoretische Informatik 1}
	\textit{Dr. Jürgen Koslowski}
	Hier geht es um formale Sprachen und Automatentheorie. Klingt theoretisch und mathelastig? Ist es auch. Nicht gleich aufgeben, wenn man in der Vorlesung nicht mitkommt, die kleinen Übungen helfen beim Verständnis und bei der Klausurvorbereitung. Sie ist regulär für das dritte Semester vorgesehen, wer es sich zutraut kann sie aber schon im ersten hören und sich damit den Stundenplan im dritten Semester ein wenig freihalten. 
\subsubsection{Mathewahlpflicht}
%	\textit{N.N.} 
Ihr müsst insgesamt zwei Module zu je fünf Credits
	im Mathe-Wahlpflichtbereich einbringen. Dabei wird eine Vorlesung im Wintersemester und zwei
	Vorlesungen im Sommersemester	angeboten:
	\begin{itemize}
	  \item Sommersemester: 
	    \begin{itemize} 
	      \item Algebra für Informatiker: Hier gehts um grundlegende
		algebraische Strukturen (Mengen, Gruppen, Monide etc). Diese sind insbesondere für die
		theoretische Informatik von großer Bedeutung.
	      \item Einführung in die Stochastik für Informatiker: Die
		Vorlesung behandelt die Grundlagen der
		Wahrscheinlichkeitstheorie (Laplace- Experimente,
		Erwartungswerte, Zufallsvariablen etc.). 
	    \end{itemize}
	  \item Wintersemester: 
	    \begin{itemize}
	      \item Einführung in die Numerik für Informatiker: Hier
		werden Verfahren zum Lösen numerischer Probleme
		behandelt. 
%	      \item Statistische Verfahren für Informatiker: Hier geht
%		es um statistische Probleme und wie man sie lösen kann.
%		Achtung: Die Vorlesung baut auf  ,,Einführung in
%		die Stochastik'' auf und setzt deren Inhalte voraus! Sie wird jedoch derzeit nicht angeboten.
	    \end{itemize}
	\end{itemize}
	Bei der Auswahl geht ihr am Besten so vor, dass ihr euch erstmal
	in alle gerade angebotenen reinsetzt und dann die behaltet, mit der ihr besser
	klarkommt. Generell gilt aber bei mathematischen Vorlesungen: Es
	gibt im Allgemeinen kein aktuelles Skript, wer nichts verpassen
	will, muss in der Vorlesung mitschreiben. Auch können die
	Hausaufgaben gerne mal umfangreicher werden, bereiten aber dafür
	sehr gut auf die Klausur vor. Dranbleiben und sich nicht
	entmutigen lassen ist alles :)
%%%%%%%%%%%%%%%%%%%%%%%%%%%%%%%%%%%%%%%%%%%%\end{multicols}