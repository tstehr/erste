\begin{multicols}{2}
\subsection{Eure Veranstaltungen im ersten Bachelor-Semester}
	Um euch einen kleinen Vorgeschmack auf die Leute zu geben, die euch im ersten Semester mit ihrem Wissen beglücken wollen, seien sie hier kurz aufgeführt:

\subsubsection{Algorithmen und Datenstrukturen}
	\textit{Prof. S\'andor Fekete}
	Diese Vorlesung vermittelt Programmiersprachenunabhängige Konzepte wie Bäume, Listen oder Stacks. Wer nicht weiß, was sich hinter diesen Begriffen verbirgt, sollte auf keinen Fall die Übungen verpassen.

\subsubsection{Programmieren 1}
	\textit{Dr. Werner Struckmann}
	Programmiert wird hier fast ausschließlich in Java. Wer keine oder nur wenig Erfahrungen mit Java gemacht hat, sollte unbedingt die kleinen Übungen bearbeiten.

\subsubsection{Lineare Algebra}
	\textit{Dr. Wolfgang Marten}
	Hier geht es um Vektoren und Matrizen, sowie ein wenig Gruppentheorie. Die übungen sind zwar nicht immer einfach, geben aber einen sehr guten Ausblick auf die Klausur.

\subsubsection{Theoretische Informatik 1}
	\textit{Dr. Jürgen Koslowski}
	(optional im 1. Semester) Hier geht es um die formale Sprachen und Automatentheorie. Klingt theoretisch und mathelastig? Ist es auch. Nicht gleich aufgeben, wenn man in der Vorlesung nicht mitkommt, die kleinen Übungen helfen beim Verständnis und bei der Klausurvorbereitung.

\subsubsection{Diskrete Mathematik}
	\textit{Prof. Arnfried Kemnitz und PD JP Bode}
	Diskrete Mathematik handelt von allem, was mit ganzen Zahlen zu tun hat: Fibbonacci-Zahlen, Primzahlen, Modulorechnung, usw. 
\end{multicols}
