	\subsection{ABC der Studentischen Selbstverwaltung}
	Alle Studierenden unsere Uni bilden zusammen die verfasste
	Studierendenschaft. Ihre Vertreter setzen sich für dich und die
	Interessen anderer Studierenden unter Anderen in folgenden
	Gremien ein:

	\subsubsection*{AStA}
		Der Allgemeine Studierenden Ausschuss ist die
		\emph{Exekutive} der Studierdendenschaftten: Er vertritt
		sie nach Außen, also zum Beispiel bei Verhandlungen um
		das Semesterticket, versorgt dich mit Informationen zu
		politischen Themen (öfter im Semester erscheint der so
		genannte \emph{AStA-Issue}) und ist einer der ersten
		Ansprechpartner für  Anliegen, die die gesamte Uni
		betreffen. Ausserdem bietet er einige Service-Angebote
		an (Kopierer, Internationaler Studierendenausweis etc.).

		\subsubsection*{Sonstige Gremien}
		Neben der verfassten Studierendenschaft gibt es an der Uni  unzählige Gremien, hier seien die wichtigsten genannt. Jedes Gremium hat eine bestimmte Besetzung, also eine definierte Anzahl von jeweils Studenten, Mitarbeitern und Professoren.

		Am relevantesten für dich ist die Studienkommission (\emph{StuKo}): Hier werden Details des Studiengangs besprochen, Probleme der Studenten geklärt und die Vergabe der Studiengebühren entschieden. In diesem Gremium herrscht ein Stimmengleichgewicht zwischen Studenten und Professoren. Das bedeutet, dass wir hier wirklich die Möglichkeit haben, aktiv in die Unipolitik einzugreifen.

		In der \emph{Informatik-Kommission} und im
		\emph{Fakultätsrat} (der außerdem noch Mitglieder aus
		den Sozial- und Wirtschaftswissenschaften, sowie der Mathematik hat) sieht es da schon schlechter aus, die Studenten stellen in beiden nur eine Minderheit der Stimmen.

		Die \emph{Berufungskommission} hat nur selten zu tun: Wann immer eine Professur besetzt werden muss, tagt sie um Kandidaten für das Amt zu finden.

		Wann immer du einen Antrag im Prüfungsamt stellt, landet
		dieser im \emph{Prüfungsausschuss}, der über ihn
		entscheidet. In diesem sind drei Professoren, ein Student und ein Mitarbeiter vertreten.

		Der \emph{Zulassungsausschuss} bearbeitet einmal im
		Semester die Master-Bewerbungen. Dabei haben wir eine
		beratende Stimme. Wir versuchen dabei, zu erreichen,
		dass es möglichst wenige  und sinnvolle Zulassungsauflagen gibt.

	\subsubsection*{Studierendenparlament}
		Eines der wichtigsten Elemente der studentischen
		Mitbestimmung ist das Studierendenparlament (Uni-Slang:
		StuPa). Es wird jedes Semester gewählt und entscheidet
		unter anderem über den studentischen Haushalt, für den
		du einen Teil des Semesterbeitrags zahlst. Außerdem werden hier Ausschüsse gewählt (Als wichtigster der \emph{Allgemeine Studierenden Ausschuss}, kurz AStA).
	\subsection*{Uniweite Vollversammlung (VV)}
		Mindestens einmal im Semester findet eine
		Vollversammlung aller Studenten statt, d.h. theoretisch
		stürmen 13500 Studenten ins Audimax. Obwohl jeder kommen
		soll, reicht schon ein winziger Bruchteil dessen, damit
		die VV beschlussfähig ist. So können hier wichtige
		Themen abgestimmt werden, die alle Studenten betreffen,
		zum Beispiel wurde die Einführung des Semestertickets
		hier beschlossen. Leider kommt meist nichtmal dieser
		Bruchteil zustande, so dass die VV dann nur Empfehlungen
		an das StuPA aussprechen kann. Wenn du informiert darüber 
		bleiben willst, was neben deinen Studiengang so an der
		Uni vor sich geht, solltest du diese Versammlungen nicht verpassen.

		\subsubsection*{Vollversammlung der Informatik}
			Was vor vielen Jahren noch regelmäßig war, ist
			zwischenzeitlich etwas eingeschlafen. Seit dem
			Sommersemester 2010 gibt es aber wieder VV's der
			Informatikstudenten, auf denen der
			Fachgruppenrat
			wichtige Informationen verkündet und einen
			breiteren Dialog sucht, als es über die
			Fachgruppentreffen oder die Mailkommunikation
			möglich ist. Hier ist die Möglichkeit für jeden,
			sich mit minimalem Aufwand in die Gestaltung des
			Studienganges einzubringen. Dabei gilt: Wenn
			20\% der Studenten anwesend sind  ist die VV offiziell beschlussfähig.
