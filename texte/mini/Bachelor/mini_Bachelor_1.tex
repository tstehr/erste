%
\end{multicols}
\section*{Das Wichtigste für den Bachelor}
  Auf detailiertere Informationen in der vollständigen Ausgabe der 1-ten wird wie folgt
  verwiesen: \mehrInfo{Abschnitt, Seite}.%%
\end{multicols}
\subsection*{Termine}
In der Tabelle steht das B für Bachelor und das M für Master.
\mehrInfo{,,Die ersten Tage'', Seite 3 \& 4}.
\begin{tabular}{|l|l|p{6.7cm}|c|c|c|}
\hline \textbf{Datum} & \textbf{Uhrzeit} & \textbf{Veranstaltung}	& \textbf{Ort} & \textbf{B} & \textbf{M} \\
\hline 26.09. – 30.09.	& 1. Tag: 10:00	 & Vorkurs Informatik
& PK 2.1		&B& \\
17.10. - 21.10.  & & & & &    \\
%\hline 11.10. – 22.10. 	&		 & Vorkurs Mathematik 								&				&B& \\
\hline Montag,  24.10. 	&  09:00 – 10:00	 & Begrüßung	durch den Präsidenten									& Stadion		&B&M\\ % TODO wirklich Audimax?
\hline 			& 10:00 – 12:00	 & Infobörse										& Altgebäude	&B&M\\
\hline   	& 10:00 – 11:00	 & Begrüßung (M) \newline durch den Studiendekan	& IZ 160	& &M\\
\hline 	& 11:30 – 13:00	 & Ersti-Frühstück									& IZ Plaza		&B&M\\
\hline 			& 13:05 – 14:00	 & Einführung  der Fachgruppe					& IZ 160		&B&M\\
\hline 			& 14:00 – 15:00	 & Begrüßung (B) \newline
durch den Studiendekan	& PK 11.3	&B& \\
\hline 			& 15:00 – 16:30	 & Erste Vorlesung
,,Programmieren 1''				& SN 19.1		&B &\\
%\hline 			& 14:00 – 15:00	 & Gemeinsames Stundenplan-Bauen					& IZ 161		& &M\\
%\hline 			& 15:00 – 16:30	 & Rundgang mit der Tutorengruppe					& SN 19.1		&B&M \\
%
%
%\hline 			& 15:00 – 15:30	 & Rundgang  \newline mit der Tutorengruppe		& IZ 150		&B&M \\
%\hline 			& 15:30 – 16:30	 & Gemeinsames Stundenplan-Bauen					& IZ 161		& B&\\
\hline Dienstag, 25.10		& 10:00 - 11:45 &
Erstsemester-Frühstück & IZ Plaza & B &M \\ 
\hline & 12:00 - 13:30 * & FG-Einführung und  \newline
Stundenbahn-Bauen & ??? * & B &\\%& ??? * & B &\\
\hline & 12:00-13:30 * & FG-Einführung und   \newline Stundenbahn-Bauen & IZ 160 * & & M\\
\hline &13:30 - 15:00 & Rundgang mit den  Tutorengruppen & IZ 160 & B
& M\\
\hline Donnerstag, 03.11 & 19:00 & Kneipentour der Fachgruppe & IZ 150 & B & M\\
\hline Mittwoch, 09.11 & 19:00 & Spieleabend der Fachgruppe & vor  150 & B & M\\
%17:00 – \ldots & Grillen / Spieleabend  mit
%der Fachgruppe (wetterabhängig :) )
%& siehe Blog		&B&M\\
%\hline 27.10.	& 10:00 – 17:00	 & Studium Generale									& Altgebäude	&B&M\\
%\hline 15.04.	& 19:00 – \ldots & Kneipentour mit der Fachgruppe					& IZ 150		&B&M\\
%\hline 10.11.	& 16:30 – \ldots & Spieleabend der Fachgruppe						& vor IZ 150	&B&M\\ 
\hline
\end{tabular} 

%%% Local Variables: 
%%% mode: latex
%%% TeX-master: "../../1-te"
%%% End: 

\subsection*{Checkliste}
% !TEX root = ../../1-te.tex

\begin{center}
\begin{tabular}{|p{3mm}|l|p{8cm}|c|c|}
\hline \checkmark 
       & \textbf{Todo}             & \textbf{Zu erledigen bis}                                  & \textbf{Seite}               & \textbf{Muss?} \\ 
\hline & BAföG beantragen          & Spätestens Ende \iftoggle{winter}{Oktober}{April}          & \pageref{todobafoeg}         & optional \\ 
\hline & Wohnsitz ummelden         & 1 Woche nach Umzug                                         & \pageref{todoummelden}       & ja \\
\hline & Mailinglisten             & So früh wie möglich                                        & \pageref{mailinglisten}      & ja \\ 
\hline & Studiengrobplanung        & Vor dem Stundenplan bauen                                  & \pageref{grob}               & ja \\
\hline & Auflagen klären           & So früh wie möglich, final: Ende 2. Semester               & \pageref{auflagen}           & nur Master \\ 
\hline & Persönlicher Stundenplan  & Siehe Terminzettel der Fachgruppe                          & \pageref{masterstundenplan}  & ja \\ 
\hline & Prüfungsbogen             & Spätestens \iftoggle{winter}{Dezember}{Mai}                & \pageref{todoanmeldung}      & ja \\ 
\hline & Prüfungsanmeldung         & 12.12.2017 - 11.01.2018, schriftlich oder online           & \pageref{todoanmeldung}      & ja \\ 
\hline & Blog abonnieren           & So früh wie möglich                                        & \pageref{fachgruppe}         & ja \\ 
\hline & Prüfungsordnung lesen     & Zu den ersten Klausuren                                    & \pageref{po}                 & ja \\ 
\hline & TUcard validieren         & Zu Beginn und zu jedem neuen Semester                      & \pageref{tucard}             & ja \\
\hline & Bibliotheksausweis        & Vor der ersten Buchausleihe                                & \pageref{todobib}            & optional \\
\hline & Stud.IP-Nachrichten weiterleiten  & Wenn man nichts verpassen möchte         & \pageref{tumails}            & optional \\
\hline
\end{tabular} 
\end{center}
\tocheck{4}{Exakte Daten Anmeldewoche einfügen, s.\url{https://www.tu-braunschweig.de/fk1/service/informatik/pa}}
%
%
\begin{multicols}{2}
\subsection*{Tutorien}
Nach dem Erstsemesterfrühstück werden die Erstsemester nach Bachelor und
Master getrennt in Tutorengruppen aufgeteilt. Diese kleinen Gruppen erkunden den Campus. Auftretende Fragen sollten dem Tutor gestellt werden. Im 
\mehrInfo{Abschnitt Tutorien, Seite 5} findet ihr sowohl mehr Informationen, als auch Bilder und Emailadressen
einiger Tutoren.%\newpage
\subsection*{Studienplanung}
Pro Semester sollten ungefähr 30 Credits erreicht werden, damit man in
6 Semestern den Bachelor und in 4 den Master besteht.
Der folgende Stundenplan für das 1. Semester und der Studienplan ist lediglich ein Vorschlag an den ihr nicht gebunden seid. Weitere Informationen:\\
\mehrInfo{Allgemeine Studienplanung ab  Seite 7}\\
\mehrInfo{ Zum Bachelor ab Seite
  18 }\\
\subsection*{Ansprechpartner}
Zunächst einmal wäre da für Vorlesungen der jeweilige Dozent zu nennen,
keiner beißt :) Auerdem könnt ihr uns als Fachgruppe unter
\nurl{fginfo@tu-bs.de}  erreichen. Außerdem treffen wir uns regelmäßig,
die Termine findet ihr unter
\nurl{http://fginfo.cs.tu-bs.de/index.php/fachgruppe/fachgruppenrat/}
Es gibt aber noch eine Menge anderer Ansprechpartner
(Studiengangskoordinatorin, Studienberatung etc):
\mehrInfo{Abschnitt Sonstiges, Seite 54}. 

\subsection*{Wichtige Links}
\begin{description}
  \item[TU-Homepage]~\\\nurl{http://tu-braunschweig.de/}
 \item[Hauptseite der Informatik]~\\\nurl{http://www.cs.tu-bs.de/}
 \item[Fachgruppenrat Informatik]~\\\nurl{http://fginfo.cs.tu-bs.de/}
 \item[Fachbereichssekretariat / Prüfungsamt]~\\
 \nurl{http://tu-braunschweig.de/fk1/service/informatik}\\
\item[Gesamtstundenplan]~\\\nurl{http://www.cs.tu-bs.de/stundenplan/}\\
\item[Vorlesungsverzeichnis]~\\\nurl{http://vorlesungen.tu-bs.de/}\\
\item[Inoffielles Forum von Studis für
  Studis]~\\\nurl{https://www.clevershit.de/}\\
\item[Modulhandbuch]~\\\nurl{http://mhb.tu-bs.de/}\\
%\item[Allgemeines Vorlesungsverzeichnis:] ~\\
%{\footnotesize\url{http://vorlesungen.tu-bs.de}}
\item[Uni-Bibliothek:] ~\\\nurl{http://www.biblio.tu-bs.de}
\item[Druckkosten:] ~\\\nurl{http://www.tu-braunschweig.de/it/services/drucken/kosten}
\item[Don't panic online] ~\\
{\footnotesize\url{http://www.tu-braunschweig.de/Medien-DB/it/dontpanic.pdf}}
\end{description}

\subsection*{Sonstiges}
Hier findet ihr Verweise zu mehr oder wichtigen Themen, auf die wir hier
aber aus Platzgründen nicht näher eingehen können:
\begin{description}
  %\item[27C3]~Das große Hackertreffen und du bist dabei\\\mehrInfo{Seite 55}
\item[Computer]~in der Informatik oder etwa nicht?\\\mehrInfo{Seite 34}
  \item[Freizeit] Es gibt ein Leben neben dem Studium\\\mehrInfo{Seite
    44} 
% TODO
%  \item[Interview] Ein Dozent erzählt \mehrInfo{}
\item[Gruppen und Unipolitik]~Bildungspolitik für alle\\\mehrInfo{Seite
  48}
\item[Lernräume] Asyl für den Prüfungszeitraum\\\mehrInfo{Seite 55}
  \item[Semesterticket] Spaß mit der Deutschen Bahn\\\mehrInfo{Seite 56}
\end{description}
