\section{Computer und so\ldots}
	\label{computer}

\begin{multicols}{2}
	\emph{Informatik hat viel mit Computern zu tun!} - Diesem (Irr-)glauben erliegen zu Anfang des Studiums einige, auch wenn sich inzwischen öfter rumspricht, dass das Studium abstrakter sein kann. Das Informatikstudium ist nicht dafür da, dir beizubringen, wie man einen Computer bedient. Somit sind diese Seiten eventuell das erste und letzte Mal,  dass dir Infos zu diesem Thema direkt vorgesetzt werden. Natürlich können wir hier nur ein paar Tipps geben und euch darauf hinweisen, wo man mehr Infos  findet.

	In Wirklichkeit hängt es von deiner Spezialisierung im Studium ab, ob  du den Computer im Studium mehr brauchen wirst als ein Student der Germanistik oder Sozialwissenschaften. Denn die einzigen Inhalte,  die jeder direkt am Rechner lernen und umsetzen muss, sind die Hausaufgaben,  die in Programmieren aufgegeben werden, sowie später noch das SEP und das Teamprojekt. Den Rest der Informatik kannst du theoretisch komplett auf dem Papier absolvieren.

	Dennoch sind Computer ein unersetzliches Werkzeug um durchs Studium zu kommen. Und je nach den von dir gewählten Modulen kann sich das oben gesagte auch ins Gegenteil verkehren, so dass du mehr Zeit vorm Rechner als im Bett verbringst. Auf den nächsten Seiten findest du dazu mehr Informationen.

% !TEX root = ../../1-te.tex

	\subsection{Wozu Computer?}
		\subsubsection{Vorlesungen Online}
			Zu den meisten Vorlesungen kannst du die Skripte im Internet finden. Für einige Vorlesungen gibt es sogar Ton- oder Videomitschnitte.

			Es gibt auch immer engagierte Studierende, die ihre Vorlesungsmitschriften online stellen. Da diese sehr wahrscheinlich in deinem Semester sind, hilft es, wenn du dich in den Vorlesungen umhörst.

		\subsubsection{Organisatorisches ohne Papier}
			Ansonsten gibt es eine Reihe von Informationen, die du vor allem im Internet findest, auch mehr und mehr Formalitäten (zum Beispiel die Prüfungsanmeldung\footnote{\verUrl{6}{https://vorlesungen.tu-bs.de}}) können dort geregelt werden. Desweiteren kannst du dir auf den Webseiten der TU Braunschweig einen individuellen Stundenplan zusammenstellen, in Erfahrung bringen, wann die nächsten Klausuren stattfinden oder das Prüfungsamt geöffnet hat\footnote{\verUrl{6}{https://www.tu-braunschweig.de/fk1/service/informatik/pa}}, lesen, was es in der Mensa zu essen gibt\footnote{\verUrl{6}{http://www.stw-on.de/braunschweig/essen/menus/mensa-1}}, offene HiWi-Stellen bei den Instituten finden\footnote{\verUrl{6}{https://www.tu-braunschweig.de/wirueberuns/stellenmarkt/wen-wir-suchen}} und vieles mehr.

		\subsubsection{Mitschreiben am PC}
			Auf den ersten Blick mag es naheliegen, sich während der Vorlesungen Notizen am Laptop anzufertigen. In der Praxis gibt es da aber eine Reihe von Problemen, vor denen wir  warnen möchten. Es hat schließlich seinen Grund, das nur rund 5\% der Studierenden in der Vorlesung am Laptop sitzen: Die meisten Tafelanschriften bestehen  aus verschachtelten Formeln, fremdartigen Buchstaben und verworrenen Zeichnungen. Diese in Echtzeit in den Laptop einzuhacken ist eine besondere Kunst, die du mit Notepad und Word gar nicht erst probieren brauchst. Eine Chance hast du vielleicht mit einem Tablet oder wenn du \LaTeX{} bereits im Schlaf beherrschst -- aber wer tut das schon zu Beginn des Studiums?

			In den Vorlesungen, in denen du nicht tafelweise abschreiben, sondern nur hier und da mal etwas notieren musst, ist ein PC schon nützlicher. Wenn du ab und zu den Vortrag der bzw. des Profs damit vergleichen möchtest, was er oder sie in das Skript geschrieben hat, kann dir der mitgebrachte Laptop unter Umständen das Ausdrucken von ein paar hundert Seiten ersparen. Du wirst aber schnell merken, dass es in praktisch keinem der Hörsäle und Seminarräume Steckdosen gibt, dir nur begrenzt Platz zur Verfügung steht und einige Profs mit technischen Geräten in der Vorlesung so ihre Probleme haben.

		\subsubsection{Hausaufgaben am PC}
			In vielen Fächern musst du regelmäßig Hausaufgaben erledigen und abgeben. Keiner erwartet von dir, dass diese mit dem PC gemacht werden, manchmal müssen sie sogar handschriftlich sein. Es hat aber auch gewisse Vorteile, sie am Computer zu schreiben (z.B. mittels \LaTeX) und dann auszudrucken.

		\subsubsection{\LaTeX}
			Bei \LaTeX handelt es sich um ein Satzsystem für wissenschaftliche Texte, wie Haus- oder Abschlussarbeiten. Erwähnenswert ist die hervorragende Unterstützung für den Satz mathematischer Formeln und, dass dabei mit Befehlen, ähnlich wie in HTML gearbeitet wird. Es gibt \LaTeX-Kurse\footnote{Angeboten z.B. durch das GITZ: \verUrl{6}{https://www.tu-braunschweig.de/it/dienste/61/6111}},
			 aber mit den Infos im Web kann man sich das auch selbst beibringen. Je eher du damit anfängst, desto weniger Probleme hast du später, wenn du damit z.B. deine Abschlussarbeit aufsetzt.

		\subsection{Computer-Pools an der Uni}
			Es ist immer nützlich zu wissen, wo man mal schnell an einen Computer kann.

			\begin{itemize}
				\item[*] Im Erdgeschoss des Altbaus gibt es auf der rechten Seite zwei Computerräume, einer weiter vorne (\textbf{PK 4.6}) und einer genau in der Ecke des Gebäudes (\textbf{PK 4.5}). Zwei weitere Räume (\textbf{PK 4.8} und die \textbf{Datenstation}) findest du im ersten Stock des Altbaus, auch wieder in der rechten Ecke. Die Rechner in \textbf{PK 4.5} und \textbf{PK 4.8} sind mit Linux ausgestattet.

				\item[*] Reichlich Computer findest du schließlich im Gauß-IT-Zentrum (GITZ) an der Hans-Sommer-Straße. Das ist der gedrungene, fast würfelförmige, dunkle Klotz hinter dem Elektrotechnik-Hochhaus (E-Tower). Hier gibt es mehrere frei zugängliche Räume mit Linux- und Windowsrechnern. Es gibt hier auch Räume für Medienbearbeitung, wo du etwa Video-Digitalisierer, ein Tonstudio und Rechner mit der Adobe Creative Suite nutzen kannst.

				\item[*] Seit 2010 stellt das IBR (Institut für Betriebssysteme und Rechnerverbund) im Raum G40 des Informatikzentrums einen Rechnerraum mit vielen, schnellen Linux-Rechnern  zur Verfügung. Zu diesem CIP-Pool (Computer-Investitions-Programm) bekommt man mit seiner y-Nummer Zutritt. Wenn man Glück hat, funktioniert sogar einer der beiden Drucker in diesem Raum, so dass man zum Drucken nicht das Informatikzentrum (IZ) verlassen muss.
			\end{itemize}

		\subsection{Der eigene Rechner}
			Wenn du trotz aller Widrigkeiten planst, dir extra für dein Studium einen (tragbaren) Rechner anzuschaffen, dann hast du hier gleich ein wenig Kaufberatung: Viel (Rechen- bzw. Grafik-)Leistung brauchst du im Studium  nur für sehr wenige spezielle Fachgebiete -- das einfachste Notebook wird also vermutlich schon reichen. Wichtiger ist vielmehr die Akkulaufzeit und die WLAN-Empfangsstärke.

		\subsubsection{Welches System?}
			Dir wird auffallen, dass zwar alle Systeme geduldet sind, aber dir Linux hier deutlich öfter über den Weg laufen wird als in der freien Wildbahn. Auch wir sind große Linux-Fans und haben deshalb ab Seite \pageref{linux} ein paar Infos dazu zusammengetragen.

			Aber trotz dieser nicht ganz unauffälligen Beeinflussung gilt: Beim Betriebssystem hast du freie Wahl. Sämtliche Software, die du für's Studium brauchen  könntest, gibt es für alle großen Systeme, meist sogar gratis. Für Linux ist eh  praktisch alles frei erhältlich, für Windows spendiert Microsoft den Studierenden auch alles außer Office (siehe Seite \pageref{msdnaa}), und auch Apple bringt dich dank Studierendenrabatte durch Bachelor und Master.
% !TEX root = ../../1-te.tex

\subsection{Gauß-IT-Zentrum}

	Das Rechenzentrum der TU-Braunschweig heißt Gauß-IT-Zentrum oder kurz GITZ. Es bietet dir eine Vielzahl an Diensten an. Manche davon kannst du nur vor Ort nutzen, also in der Hans-Sommer-Str. 65, direkt hinter dem ,E-Tower'. 
	
	Andere Dienste sind auch in den Außenstellen, wie z.B. im
	Altgebäude zu finden und das allermeiste lässt sich über das Netz an der gesamten Uni oder sogar weltweit in Anspruch nehmen.

\subsubsection{GITZ-Account}
\label{todogitz}
	Unser Rechenzentrum, das Gauß-IT-Zentrum, stellt  diverse Dienste zur Vefügung, wovon manche quasi lebenswichtig sind, andere eher nebensächlich. Aber für all diese Dienste brauchst du eine GITZ-Account-Nummer und ein Passwort. Diese so genannte y-Nummer ist nicht das gleiche wie eure Immatrikulationsnummer. In der Regel bekommt man schon vor Semesterbeginn eine Nummer und ein vorläufiges Passwort per Post zugesendet. Dieses Passwort brauchst du dir nicht merken, denn man kann es nur verwenden, um  sich ein richtiges Passwort für die spätere Verwendung auszusuchen. Das sollte man schnellstmöglichst erledigen, da man sonst die Dienste des GITZ (z.B: WLAN, die Pool-Rechner etc) nicht nutzen kann. 
	
	Es kann auch passieren, dass du den besagten Brief vom GITZ  gar nicht bekommst, dann gehst du einfach selbst zum GITZ in die Hans-Sommer-Straße und besorgst dir dort einen. Keine Sorge, das passiert halt ab und zu, ist aber nicht weiter schlimm.

	\subsubsection{Emailadresse}
		Zusammen mit eurem GITZ-Account bekommt ihr auch ein neues Email-Postfach mit bis zu drei Adressen (y00000000@tu-bs.de, vorname.nachname@tu-bs.de, v.nachname@tu-bs.de). Leider kommt es dabei manchmal zu Problemen, also nicht wundern, wenn euch Emails mal mit kleiner Verzögerung erreichen. 

	\subsubsection{WLAN}
		\label{wlan}
		WLAN wird vom Rechenzentrum in vielen Hörsälen (wie dem \textbf{Audimax} und \textbf{SN19.1}), im IZ, in der Universitätsbibliothek (UB), der Mensa und im GITZ angeboten. Notebookbesitzer finden auf folgender Webseite alle notwendigen Informationen, um das \emph{eduroam} nutzen zu können. \verUrl{4}{http://www.tu-braunschweig.de/it/dienste/11/1106}

		Das \emph{eduroam} ist ein international standardisierter Zugang, der an vielen europäischen Hochschulen funktioniert. Einmal eingerichtet kannst du also mit deinen TU-BS-Zugangsdaten problemlos an anderen Unis surfen.

		Die Anleitungen der TU-Braunschweig werden dir nahelegen, eine spezielle Software nachzuinstallieren. Es geht aber für alle aktuellen Betriebssysteme auch ohne, also nur mit Boardmitteln -- um herauszufinden wie, schau einfach im Netz nach, was andere Unis zu \emph{eduroam} zu sagen haben. Für Windows XP (und eng verwandte Versionen) bietet z.B. die Uni Graz eine schöne Anleitung.

		Wer etwas schneller unterwegs sein will (oder wessen Empfang überhaupt nicht ausreicht), dem sei das normale Ethernet ans Herz gelegt. Ein Kabel dazu musst du dir selbst mitbringen. Dosen zum Anschließen gibt es in der Uni-Bibliothek (z.T. versteckt unter runden Klappen im Boden, z.T. an der Fensterseite freiliegend), dem Informatik-Zentrum, sowie einigen Rechnerräumen im Altgebäude und Rechenzentrum.

\begin{multicols}{2}
\subsection{Linux}
	\label{linux}
	Als Informatiker befasst man sich oft mit abstrakten und allgemeinen Konzepten, die unabhängig von konkreten Betriebssystemen gültig sind. Aber sobald man sich an einen Rechner setzt, hat man es dann doch mit einem konkreten System zu tun, und innerhalb der Rechnerpools an der Uni ist dies meist die eine oder andere Linux-Version. Du wirst also im Studium nicht drum herum kommen, etwas Erfahrung damit zu sammeln.

	Auf deinem eigenen Rechner kannst du natürlich machen, was immer du möchtest, aber viele von uns bevorzugen auch dort Linux oder ein anderes Unix-artiges System. Der Umstieg ist gar nicht so schwer wie man denkt bzw. wie er vor 10 Jahren mal war, und dank Live CDs, Dual Boot und Virtualisierung kannst du sogar Linux und dein bisheriges System parallel laufen lassen und somit ganz unverbindlich reinschnuppern.

	\subsubsection{Einstiegshilfen}
		Falls du mit Linux bisher keine Erfahrung hast, könnte der Studienbeginn der passende Zeitpunkt sein. Die Fachgruppe veranstaltet von Zeit zu Zeit Linux-Installationsparties die dir beim Einstieg helfen. Wenn dann im Alltag irgendein Problem auftritt, ist der nächste Linux-Guru meist nur wenige Meter entfernt.

		Auch wenn du nocht nicht 100\% sicher bist, wohin die Reise geht, solltest du also vor dem Kauf eines neuen Rechner sicherheitshalber checken, ob die Hardware Linux-Kompatibel ist.

	\subsubsection{SSH - Zugriff aus der Ferne}
		Um vom heimischen PC aus Zugriff auf deinen Uniaccount zu haben, kannst du von Linux aus ssh benutzen. Für Windowsbenutzer gibt es zwei nette kleine Tools, Putty und WinSCP. Deinen Uniaccount erreichst du über den Server \texttt{rzstudio.rz.tu-bs.de}.

		\begin{description}
			\item[Putty] stellt dir eine Shell auf dem UNIX"~Rechner bereit. Damit kannst du so auf deinem Rechner arbeiten, als würdest du direkt auf dem Server arbeiten (tust du ja auch). Um auch grafische Programme starten zu können, musst du noch einen X"~Server für Windows installieren, z.B. X-Deep32.
			\item[WinSCP] ist ein Tool, das einem FTP"~Client ähnelt. Mit diesem kannst du Dateien von und zu deinem Uniaccount kopieren. Der Vorteil ist, dass die Übertragung verschlüsselt ist und Passwörter somit nicht abgehört werden können.
		\end{description}

		Zu allen in diesem Text angesprochenen und noch zu vielen anderen Computerproblemen mehr gibt es Informationen im Heft \emph{Don't Panic}, das kostenlos im Rechenzentrum erhältlich ist. Nimm es dir gleich mit, wenn du deine y"~Nummer beantragst.

	\subsubsection{Linux-Bezug an der TU-BS}
		Fast alle Linux-Distributionen und Softwarepakete für Linux sind freie Software und somit kostenlos erhältlich.

		Für Studierende mit Breitband-Internetzugang sind vermutlich die diversen Mirror-Server an der Uni interessant. Hier stehen die größeren Distributionen bereit:
	  
		\begin{description}
			\item[\url{ftp://ftp.rz.tu-bs.de/}]~\\Enthält Openoffice-, Mozilla-, Gentoo-, Slackware- und Ubuntumirror, CCC Vorträge
			\item[\url{ftp://debian.tu-bs.de/}]~\\Debian-, Kanotix- und Knoppixmirror
			\item[\url{ftp://ftp.ibr.cs.tu-bs.de/}]~\\Mehr CCC Vorträge, diverse freie Software (größtenteils für Unix/Linux)
		\end{description}

		Für Studierende ohne breitbandigen Netzzugang sind sicherlich die CDs nützlich, die sich jede/r im IT Service-Desk\footnote{http://www.tu-braunschweig.de/it/service-desk} im Gauß-IT-Zentrum, \textbf{Raum 017}, ausleihen kann. Dort stehen eigentlich immer die neusten Versionen von SuSE, Mandrake, Fedora, Gentoo, Debian und Knoppix sowie diverse ältere Distributionen zur Verfügung. Dank eines DVD-Brenners können inzwischen auch --~soweit vorhanden (SuSE, Knoppix)~-- die DVD-Versionen verliehen werden. Auf der sicheren Seite ist, wer vorher einen Abholtermin vereinbart, damit die gewünschte Distribution garantiert greifbar ist: 0531/391-5555.
\end{multicols}
\subsection{Microsoft Academic Alliance}
	\label{msdnaa}
	Die TU hat  2003 eine Campuslizenz\footnote{\verUrl{0}{http://msdn.microsoft.com/en-us/default.aspx}} von Microsoft erworben, in deren Rahmen du Microsoftprodukte kostenlos beziehen kannst.\\ 
	Zur Auswahl stehen die meisten Betriebssysteme, Entwicklungswerkzeuge und diverse Serversoftware\footnote{\sloppy Eine komplette Liste der Software findet sich unter \verUrl{0}{http://msdn.microsoft.com/en-us/subscriptions/downloads/default.aspx}}. Die Office-Suite ist explizit \textbf{nicht} enthalten.

	Die Software darf zu nicht-kommerziellen Zwecken in Forschung und Lehre eingesetzt werden, jedoch keine Infrastrukturaufgaben erfüllen\footnote{Die Nutzungsbedingungen sind nachzulesen unter \verUrl{0}{https://www.dreamspark.com/}}. Infos gibt es unter \verUrl{0}{https://www.tu-braunschweig.de/it/service-interaktiv/software/doku/msdn-aa}.

	Du brauchst ein laufendes Windows, um Software (also auch
	Windows selbst) herunterzuladen. Du kannst  bei den Operateuren
	im Rechenzentrum in \textbf{Raum 015} eine Windows-CD gegen eine Schutzgebühr von ca. 10 Euro erwerben, die übrige Software kannst du dort ausleihen oder unter \verUrl{0}{https://www.tu-braunschweig.de/it/service-interaktiv/software/doku/msdn-aa} downloaden.

% !TEX root = ../../1-te.tex

\subsection{Elektronisch informiert}
	\label{elekinf}
	Die wichtigsten Aufgaben der Studierenden sind der Besuch von Lehrveranstaltungen, Zeitmanagement für Studium und Freizeit und Informationsbeschaffung. In diesem Artikel geht es um den letzten Punkt. Da wir nun mal Informatik studieren, soll die Informationsbeschaffung über das Internet erfolgen.

	\subsubsection*{Mailinglisten}
	\label{mailinglisten}
		Die wichtigste Mailingliste für Informatikstudierende ist die Liste \textbf{cs-studs}. Sie ist \emph{die} Informationsquelle. Hier werden Ankündigungen zu Lehrveranstaltungen gemacht, die Fachgruppe kündigt hier Spiele- und Grillabende an und es gibt oft Angebote zu Hiwistellen oder offenen Teamprojekten, Bachelorarbeiten etc. und selbstverständlich ist dies auch ein guter Ort, um Fragen zum Studium loszuwerden.

		Wer längere Diskussionen sucht, kann diese auf der Liste \textbf{cs-studs-discuss} finden bzw. führen. Diese Liste ist noch relativ neu und damit liegt es auch an euch, ihr Leben einzuhauchen.

		Da bei den Wirtschaftsinformatikern oftmals auch informatikrelevante Themen diskutiert werden, lohnt sich möglicherweise auch ein Blick in \textbf{winfo-studs}. 
		Wer an Stellenangeboten und Werbung aus der freien
		Wirtschaft interessiert ist, sollte Mailingliste
		\textbf{firmenkontakt} abonieren. Die
		Informatik-Kolloquien, das sind Vorträge von
		üblicherweise externen Referent/innen zu Informatik-Themen,
		werden auf der Mailingliste \textbf{kolloq} angekündigt.
		Unter
		\verUrl{4}{https://mail.ibr.cs.tu-bs.de/mailman/listinfo/}
		findest du eine umfassende Liste der angebotenen Mailinglisten in der Informatik.

	\subsubsection*{IRC Channel}
		Im Freenode IRC Chat (\verUrl{4}{http://freenode.net}) gibt es den Channel \url{###cs-studs}. Hier sind immer ein paar BraunschweigerInnen und große Teile der Fachgruppe online. Die Gesprächsthemen haben (im weitesten Sinne ;) mit dem Studium zu tun.

	% \subsubsection*{Clevershit}

	% 	Auf jeden Fall einen Besuch wert und eine gute Hilfe bei allem, was das Studium betrifft, ist das von Studierenden ins Leben gerufene Portal \mbox{\verUrl{2}{https://clevershit.de/}}.

	% 	Dieses von Studierenden für Studierende erstellte und geführte Plattform bietet eine gute Anlaufstelle für Fragen jeglicher Art. Es gibt eine Materialsammlung zu allen Veranstaltungen der ersten Semester.

\subsubsection*{Sonstige Informationen}
	\begin{description}
		\item[Allgemeines Vorlesungsverzeichnis:] ~\\
			{\footnotesize\verUrl{4}{http://vorlesungen.tu-bs.de}}
		\item[Uni-Bibliothek:] ~\\
			{\footnotesize\verUrl{4}{http://www.biblio.tu-bs.de}}
		\item[Druckkosten:] ~\\
			{\footnotesize\verUrl{4}{https://www.tu-braunschweig.de/it/service-interaktiv/druckkosten}}
		\item[Don't Panic online] ~\\
			{\footnotesize\verUrl{4}{http://www.tu-braunschweig.de/Medien-DB/it/dontpanic.pdf}}
	\end{description}

\end{multicols}
