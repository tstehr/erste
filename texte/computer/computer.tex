% !TEX root = ../../1-te.tex

	\subsection{Wozu Computer?}
		\subsubsection{Vorlesungen Online}
			Zu den meisten Vorlesungen kannst du die Skripte im Internet finden. Für einige Vorlesungen gibt es sogar Ton- oder Videomitschnitte.

			Es gibt auch immer engagierte Studierende, die ihre Vorlesungsmitschriften online stellen. Da diese sehr wahrscheinlich in deinem Semester sind, hilft es, wenn du dich in den Vorlesungen umhörst.

		\subsubsection{Organisatorisches ohne Papier}
			Ansonsten gibt es eine Reihe von Informationen, die du vor allem im Internet findest, auch mehr und mehr Formalitäten (zum Beispiel die Prüfungsanmeldung\footnote{\verUrl{7}{https://vorlesungen.tu-bs.de}}) können dort geregelt werden. Desweiteren kannst du dir auf den Webseiten der TU Braunschweig einen individuellen Stundenplan zusammenstellen, in Erfahrung bringen, wann die nächsten Klausuren stattfinden oder das Prüfungsamt geöffnet hat\footnote{\verUrl{7}{https://www.tu-braunschweig.de/fk1/service/informatik/pa}}, lesen, was es in der Mensa zu essen gibt\footnote{\verUrl{7}{http://www.stw-on.de/braunschweig/essen/menus/mensa-1}}, offene HiWi-Stellen bei den Instituten finden\footnote{\verUrl{7}{https://www.tu-braunschweig.de/wirueberuns/stellenmarkt/wen-wir-suchen}} und vieles mehr.

		\subsubsection{Mitschreiben am PC}
			Auf den ersten Blick mag es naheliegen, sich während der Vorlesungen Notizen am Laptop anzufertigen. In der Praxis gibt es da aber eine Reihe von Problemen, vor denen wir  warnen möchten. Es hat schließlich seinen Grund, das nur rund 5\% der Studierenden in der Vorlesung am Laptop sitzen: Die meisten Tafelanschriften bestehen  aus verschachtelten Formeln, fremdartigen Buchstaben und verworrenen Zeichnungen. Diese in Echtzeit in den Laptop einzuhacken ist eine besondere Kunst, die du mit Notepad und Word gar nicht erst probieren brauchst. Eine Chance hast du vielleicht mit einem Tablet oder wenn du \LaTeX{} bereits im Schlaf beherrschst -- aber wer tut das schon zu Beginn des Studiums?

			In den Vorlesungen, in denen du nicht tafelweise abschreiben, sondern nur hier und da mal etwas notieren musst, ist ein PC schon nützlicher. Wenn du ab und zu den Vortrag der bzw. des Profs damit vergleichen möchtest, was er oder sie in das Skript geschrieben hat, kann dir der mitgebrachte Laptop unter Umständen das Ausdrucken von ein paar hundert Seiten ersparen. Du wirst aber schnell merken, dass es in praktisch keinem der Hörsäle und Seminarräume Steckdosen gibt, dir nur begrenzt Platz zur Verfügung steht und einige Profs mit technischen Geräten in der Vorlesung so ihre Probleme haben.

		\subsubsection{Hausaufgaben am PC}
			In vielen Fächern musst du regelmäßig Hausaufgaben erledigen und abgeben. Keiner erwartet von dir, dass diese mit dem PC gemacht werden, manchmal müssen sie sogar handschriftlich sein. Es hat aber auch gewisse Vorteile, sie am Computer zu schreiben (z.B. mittels \LaTeX) und dann auszudrucken.

		\subsubsection{\LaTeX}
			Bei \LaTeX handelt es sich um ein Satzsystem für wissenschaftliche Texte, wie Haus- oder Abschlussarbeiten. Erwähnenswert ist die hervorragende Unterstützung für den Satz mathematischer Formeln und, dass dabei mit Befehlen, ähnlich wie in HTML gearbeitet wird. Es gibt \LaTeX-Kurse\footnote{Angeboten z.B. durch das GITZ: \verUrl{7}{https://www.tu-braunschweig.de/it/dienste/61/6111}},
			 aber mit den Infos im Web kann man sich das auch selbst beibringen. Je eher du damit anfängst, desto weniger Probleme hast du später, wenn du damit z.B. deine Abschlussarbeit aufsetzt.

		\subsection{Computer-Pools an der Uni}
			Es ist immer nützlich zu wissen, wo man mal schnell an einen Computer kann.

			\begin{itemize}
				\item[*] Im Erdgeschoss des Altbaus gibt es auf der rechten Seite zwei Computerräume, einer weiter vorne (\textbf{PK 4.6}) und einer genau in der Ecke des Gebäudes (\textbf{PK 4.5}). Zwei weitere Räume (\textbf{PK 4.8} und die \textbf{Datenstation}) findest du im ersten Stock des Altbaus, auch wieder in der rechten Ecke. Die Rechner in \textbf{PK 4.5} und \textbf{PK 4.8} sind mit Linux ausgestattet.

				\item[*] Reichlich Computer findest du schließlich im Gauß-IT-Zentrum (GITZ) an der Hans-Sommer-Straße. Das ist der gedrungene, fast würfelförmige, dunkle Klotz hinter dem Elektrotechnik-Hochhaus (E-Tower). Hier gibt es mehrere frei zugängliche Räume mit Linux- und Windowsrechnern. Es gibt hier auch Räume für Medienbearbeitung, wo du etwa Video-Digitalisierer, ein Tonstudio und Rechner mit der Adobe Creative Suite nutzen kannst.

				\item[*] Seit 2010 stellt das IBR (Institut für Betriebssysteme und Rechnerverbund) im Raum G40 des Informatikzentrums einen Rechnerraum mit vielen, schnellen Linux-Rechnern  zur Verfügung. Zu diesem CIP-Pool (Computer-Investitions-Programm) bekommt man mit seiner y-Nummer Zutritt. Wenn man Glück hat, funktioniert sogar einer der beiden Drucker in diesem Raum, so dass man zum Drucken nicht das Informatikzentrum (IZ) verlassen muss.
			\end{itemize}

		\subsection{Der eigene Rechner}
			Wenn du trotz aller Widrigkeiten planst, dir extra für dein Studium einen (tragbaren) Rechner anzuschaffen, dann hast du hier gleich ein wenig Kaufberatung: Viel (Rechen- bzw. Grafik-)Leistung brauchst du im Studium  nur für sehr wenige spezielle Fachgebiete -- das einfachste Notebook wird also vermutlich schon reichen. Wichtiger sind vielmehr Akkulaufzeit und physischen Eigenschaften wie Größe, Gewicht und Robustheit, wenn du das Gerät täglich mit dir herumtragen willst.

		\subsubsection{Welches System?}
Als Informatiker befasst man sich oft mit abstrakten und allgemeinen Konzepten, die unabhängig von konkreten Betriebssystemen gültig sind. Aber sobald man sich an einen Rechner setzt, hat man es dann doch mit einem konkreten System zu tun, und innerhalb der Rechnerpools an der Uni ist dies meist die eine oder andere Linux-Version. Du wirst also im Studium nicht drumherum kommen, etwas Erfahrung damit zu sammeln.

	Auf deinem eigenen Rechner kannst du natürlich machen, was immer du möchtest, aber viele von uns bevorzugen auch dort Linux oder ein anderes Unix-artiges System. Der Umstieg ist gar nicht so schwer wie man denkt bzw. wie er vor 10 Jahren mal war, und dank Live-CDs, Dual Boot und Virtualisierung kannst du sogar Linux und dein bisheriges System parallel laufen lassen und somit ganz unverbindlich reinschnuppern.

        \subsubsection{Linux-Installparty}
	Die Fachgruppe bietet im Rahmen der O-Woche eine Linux-Installparty an, auf der du unter Anleitung und mit Unterstützung von erfahrenen Linux-Nutzern dein eigenes Linux installieren kannst. Sie findet dieses Semester am 03. April um 15:00 Uhr in IZ 161 statt. \tocheck{7}{Wann findet die Linux-Install-Party dieses Semester statt?}
        Für weiterführende Workshops, Hilfe zur Selbsthilfe und Spaß mit freier Software wollen wir anschließend durch eine regelmäßig stattfindenden Veranstaltung \emph{How To Linux} organisieren. Genauere Informationen zu beiden Events findest du in unserem Wiki\footnote{\verUrl{7}{https://wiki.fginfo.tu-bs.de/doku.php?id=events:linux}}.

        \subsubsection{Anwendungssoftware}
			Aber trotz dieser nicht ganz unauffälligen Beeinflussung gilt: Beim Betriebssystem hast du freie Wahl. Sämtliche Software, die du für's Studium brauchen  könntest, gibt es für alle großen Systeme, meist sogar gratis. Für Linux ist eh  praktisch alles frei erhältlich, für Windows spendiert Microsoft den Studierenden auch alles außer Office (siehe Seite \pageref{msdnaa}), und auch Apple bringt dich dank Studierendenrabatte durch Bachelor und Master.
