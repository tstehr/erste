% !TEX root = ../../1-te.tex

\subsection{Microsoft Imagine}
	\label{msdnaa}
	Die TU besitzt eine Campuslizenz von Microsoft, in deren Rahmen du nahezu 1000 verschiedene Produkte kostenlos beziehen kannst.\\
	Zur Auswahl stehen die meisten Betriebssysteme, Entwicklungswerkzeuge und diverse Serversoftware. Die Office-Suite ist explizit \textbf{nicht} enthalten.

	Die Software darf zu nicht-kommerziellen Zwecken in Forschung und Lehre eingesetzt werden, jedoch keine Infrastrukturaufgaben erfüllen. 
Infos gibt es unter \verUrl{6}{https://www.tu-braunschweig.de/it/service-interaktiv/software/doku/msdn-aa}.

	Du brauchst ein laufendes Windows, um Software (also auch Windows selbst) herunterzuladen. Alternativ kannst du bei den Operateuren im Rechenzentrum in \textbf{Raum 015} eine Windows-DVD gegen eine Schutzgebühr von 10 Euro erwerben, die übrige Software kannst du dort ausleihen oder von der Website downloaden.
