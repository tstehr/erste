Dabei soll nicht
verschwiegen werden, dass damit auch eine große Verantwortung
einhergeht. Dadurch, dass du zu nichts verpflichtet bist, könntest du
jetzt den Eindruck gewonnen haben, dass du nichts tun musst. Das ist
falsch! ,,Dich zwingt niemand etwas zu tun.'' heißt noch lange 
	nicht ,,freiwillig'' oder ,,Ich muss nur für die Klausur lernen,
	sonst habe ich frei!'' Oft heißt es stattdessen: ,,Es
	interessiert niemanden, ob du was tust. Du bist selbst dafür
	verantwortlich!''\\\\ 
	Um Prüfungen letzlich erfolgreich zu bestehen, Punkte zu sammeln und
	am Ende einen Abschluss zu bekommen, muss also du den Stoff
	irgendwie erlernen, Die  Frage lautet also: ,,Wie gehe ich
	damit um? Wie lerne ich am Besten?''\\\\	
	Schauen wir uns dazu mal die Lehrveranstaltungen an: In ihnen
	erlernst du die theoretischen Grundlagen des Faches und bekommst
	dazu anhand der Übungen Beispiele, wie du bestimmte Probleme
	praktisch lösen kannst.
	 Dieser Stoff 	wird dir frühestens in der Klausur und danach im weiteren
	Studium begegnen, sowie später im Beruf begegnen.  \\\\	
	Also muss es dein Ziel sein, den Stoff zu verstehen, um
	dieses Wissen später anwenden zu können, egal ob du in der
	Veranstaltung warst, oder nicht. Auch wenn die Versuchung groß ist,
	eine langweilige Vorlesung oder Übung einfach zu schwänzen,
	solltest du darüber nochmal nachdenken. Alles, was du in der Uni
	verpasst,  musst du dir komplett selber  erarbeiten
	(z.B. durch Bücher oder per Internet). Also schaue immer, ob das Besuchen 
	der Veranstaltung plus	Nacharbeit nicht doch das keinere Übel ist. \\\\
	Gerade im ersten Semester, wo du deinen persönlichen Lernstil
	noch finden musst solltet du das nicht vergessen.
	\\\\
	Trotzdem: Genieße deine neuen Freiheit, aber nutze sie weise, bevor sie
	zum Fluch wird. :)
\\\\
% %	Denn dass es keinen Zwang hier gibt, heißt leider nicht, dass
% %\begin{comment}
% %  \\ \\
% %\subsection{Lernen an der Uni-Kein Zwang aber auch nicht freiwillig}
% 	Das Leben und Lernen an der Uni ist sehr spannend. Es bieten sich viele Möglichkeiten, das Studium individuell zu gestalten, nach Interessen zu wählen und schließlich den erwünschten Abschluss zu erhalten. Ein Student genießt große Freiheiten. Aus großen Freiheiten ergibt sich aber auch eine große Verantwortung. Wie das zusammenhängt und welche Gefahren daraus resultieren, soll hier einmal kurz aufgearbeitet werden.
	
% 	Grundsätzlich gilt an der Uni zunächst, dich zwingt niemand irgendetwas zu machen. 
% 	Vorlesungen können besucht werden, müssen aber nicht. Hausaufgaben können gemacht werden, 
% 	müssen aber nicht. Prüfungen können abgelegt werden, müssen aber
% 	nicht.
% 	Dieses Konzept spiegelt eine gewisse Scheinfreiwilligkeit wieder, die es aber gar nicht ist. 
% 	Der spannende Unterschied ist der folgende: ,,Dich zwingt
% 	niemand etwas zu tun.'' heißt noch lange 
% 	nicht ,,freiwillig'' oder ,,Ich muss nur für die Klausur lernen,
% 	sonst habe ich frei!'' \\\\ 
% 	Um Prüfungen erfolgreich zu bestehen, Punkte zu sammeln und
% 	schlussendlich einen Abschluss zu bekommen, muss also du den Stoff
% 	irgendwie erlernen, Die spannende Frage ist aber: ,,Wie gehe ich
% 	damit um? Wie lerne ich am Besten?''\\\\	
% 	Schauen wir uns dazu mal die Lehrveranstaltungen an: In ihnen
% 	erlernst du die theoretischen Grundlagen des Faches und bekommst
% 	dazu anhand der Übungen Beispiele, wie du bestimmte Probleme
% 	praktisch löst. 
% 	 Dieser Stoff
% 	wird dir frühestens in der Klausur und danach im weiteren
% 	Studium begegnen.  \\\\	
% 	Also muss es dein Ziel sein, den Stoff zu verstehen, um
% 	dieses Wissen später anwenden zu können, egal ob du in der
% 	Veranstaltung warst, oder nicht. Auch wenn die Versuchung groß ist,
% 	eine langweilige Vorlesung oder Übung einfach zu schwänzen,
% 	solltest du darüber nochmal nachdenken. Alles, was du in der Uni
% 	verpasst,  musst du dir selber  erarbeiten
% 	(z.B. durch Bücher oder per Internet). Also schaue immer, ob das Besuchen 
% 	der Veranstaltung plus	Nacharbeit nicht doch das keinere Übel ist. \\\\
% 	Gerade im ersten Semester, wo du deinen persönlichen Lernstil
% 	noch finden musst solltet du das nicht vergessen.
% 	\\\\
% 	Genieße deine neuen Freiheit, aber nutze sie weise, bevor sie
% 	zum Fluch wird. :)\\\\
% %      \end{comment}
% %:wLernen freiwillig ist und man nichts tun muss. 
% %	doch das kleinere ÜKlar, manche Vorlesungen sind gähnend langweilig, manche Vorlesungen sind viel zu theoretisch und manchen Dozenten kann einfach nicht zugehört werden. Das sind alles Gründe, irgendwann nicht mehr in die Vorlesung zu gehen, aber dann fehlt eben ein wichtiger Teil des Lernens. "Ich kann doch ein oder zwei Bücher lesen und mir das Wissen selber aneignen." Ja, das ist richtig, das kannst du machen. Für einige mag dies tatsächlich der bessere Weg sein, aber im großen und ganzen ist dies viel mühsamer als die Vorlesung zu besuchen.
	
% %	Was heißt das jetzt genau? \\
% %	Das heißt eigentlich nur eines: Lass dich von deinen neu gewonnen Freiheiten nicht daran hindern, erfolgreich zu studieren. Besuche lieber einmal mehr die Vorlesung als das eine mal zu wenig. Gerade in den ersten Semestern ist dies wärmstens von uns empfohlen. \\
% %	Es gilt immer der Grundsatz $ kein Zwang \neq freiwillig $ (dt: Kein Zwang ist nicht gleich freiwillig).
	
\begin{center}
\includegraphics[totalheight=6cm]{bilder/XKCD/priorities}
\end{center}
