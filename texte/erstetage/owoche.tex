% !TEX root = ../../1-te.tex

\subsection{Wichtige Termine am Anfang des Studiums}

\tocheck{8}{Termine für aktuelle O-Woche einfügen}

Wir möchten den Start an der TU Braunschweig so gut wie möglich begleiten. Daher wird es zu Beginn des Semesters wieder Begrüßungs- und Einführungsveranstaltungen geben. Eine Übersicht über die Termine findest du auch auf der Rückseite dieser Zeitschrift. Das System der Raumnummern ist auf Seite \pageref{campuskarte} erklärt.

Bis zum Semesterstart können sich einzelne Termine noch ändern. Den ganz aktuellen Stand gibt es online unter \verUrl{8}{https://fginfo.tu-braunschweig.de/ersti/}.

\renewcommand{\labelitemi}{$\bullet$}
\renewcommand{\labelitemii}{$\bullet$}
\renewcommand{\labelitemiii}{$\bullet$}
\renewcommand{\labelitemiv}{$\bullet$}

\begin{itemize}
    \item Montag, 14. Oktober
        \begin{itemize}
            \item 10:00 Uhr: Gemeinsames Frühstück (Plaza, IZ 1.OG) \textbf{Bitte eigenes Geschirr, Becher (für Kaffee o.ä.) und Besteck mitbringen}
            \item 11:00 Uhr: Info-Vortrag \& Stundenplanbau-Workshop (IZ 160)
            \item \textit{13:00 - 17:00 Uhr: Vorkurs}
        \end{itemize}
    \item Dienstag, 15. Oktober
        \begin{itemize}
            \item \textit{9:00 - 13:00 Uhr: Vorkurs}
            \item 13:00 Uhr: Mensabesuch (Mensa 1)
            \item 14:00 Uhr: Campustour (Treffen Fachgruppenraum Informatik, Foyer IZ)
        \end{itemize}
    \item Mittwoch, 16. Oktober
        \begin{itemize}
            \item \textit{9:00 - 13:00 Uhr: Vorkurs}
            \item 14:00 Uhr: Nachmittagsprogramm, du kannst zwischen mehreren Optionen auswählen
            \item 17:00 Uhr: Linux-Install-Party (IZ 161)\footnote{Weitere Infos: \verUrl{8}{https://wiki.fginfo.tu-bs.de/doku.php?id=events:linux}}
        \end{itemize}
    \pagebreak
    \item Donnerstag,  17. Oktober
        \begin{itemize}
            \item \textit{9:00 - 13:00 Uhr: Vorkurs}
            \item 14:00 Uhr: Nachmittagsprogramm, du kannst zwischen mehreren Optionen auswählen
            \item 19:00 Uhr: Kneipentour (Start Haupteingang der Mensa 1)
        \end{itemize}
    \item Freitag, 18. Oktober
        \begin{itemize}
            \item \textit{11:00 - 16:00 Uhr: Vorkurs}
        \end{itemize}
    \item Montag, 21. Oktober
        \begin{itemize}
            \item \textit{9:00 Uhr: Zentrale Begrüßung aller Erstsemester (Eintracht-Stadium)}
            \item \textit{10:30 – 12:00 Uhr: Infobörse „Studium ist mehr“ (Foyer von Altgebäude und Audimax)}
            \item \textit{11:30 – 13:00 Uhr: Vorlesung: Lineare Algebra (PK 2.2)}
            \item \textit{13:15 - 14:15 Uhr: Erstibegrüßung der Informatik (PK 2.1)}
        \end{itemize}
    \item Dienstag, 22. Oktober
        \begin{itemize}
            \item \textit{13:15 - 14:45 Uhr: Übung: Diskrete Mathematik (PK 2.2)}
        \end{itemize}
    \item Mittwoch, 23. Oktober
        \begin{itemize}
            \item \textit{9:45 - 11:15 Uhr: Vorlesung: Diskrete Mathematik (PK 11.2)}
            \item \textit{15:00 – 16:30 Uhr: Vorlesung: Lineare Algebra für Informatiker (PK 2.2)}
            \item 19:00 Uhr: Analoger Spieleabend (Flur vor IZ 150)
        \end{itemize}
    \item Donnerstag, 24. Oktober
        \begin{itemize}
            \item \textit{keine Termine}
        \end{itemize}
    \item Freitag, 25. Oktober
        \begin{itemize}
            \item \textit{9:45 – 11:15 Uhr: Übung: Lineare Algebra für Informatiker (PK 2.2)}
            \item \textit{13:00 Uhr: Treffen der Medizininformatiker (IZ 404) \textbf{Nur für Medizininformatikstudierende. Um Anmeldung wird gebeten, wenn du Medizininformatik studierest und teilnehmen willst, schicke eine Mail an ute.zeisberg@plri.de}}
            \item 14:30 Uhr: Treffen Abfahrt Erstifahrt (Foyer IZ, Mühlenpfordstraße 23)
        \end{itemize}
    \item Ersti-Wochenende
        \begin{itemize}
            \item Wann? \textit{25. – 27. Oktober}
            \item Wo? Naturfreundehaus Eichsfelder Hütte (St. Andreasberg)
            \item Was? Lerne deine Mitstudierenden kennen, habe Spaß :)
            \item Finanzierung? Größtenteils aus Studienqualitätsmitteln, dazu ein kleiner Selbstkostenbeitrag
            \item Fristen: \textit{Anmeldung und Bezahlung des Selbstkostenbeitrags bis 23. Oktober}
            \item Du kannst dich unter folgender Adresse für die Erstifahrt anmelden: \verUrl{8}{https://tickets.fginfo.tu-bs.de/fginfo/efws19/}
        \end{itemize}
    \end{itemize}
