% !TEX root = ../../1-te.tex

\subsection{Wichtige Termine am Anfang des Studiums}

\tocheck{6}{Termine für aktuelle O-Woche einfügen}

Wir möchten den Start an der TU Braunschweig so gut wie möglich begleiten. Daher wird es zu Beginn des Semesters wieder Begrüßungs- und Einführungsveranstaltungen geben. Eine Übersicht über die Termine findest du auch auf der Hinterseite dieser Zeitschrift. Bis zum Semesterstart können sich einzelne Termine noch ändern. Den ganz aktuellen Stand gibt es online unter \verUrl{6}{https://fginfo.cs.tu-bs.de/ersti}.

\renewcommand{\labelitemi}{$\bullet$}
\renewcommand{\labelitemii}{$\bullet$}
\renewcommand{\labelitemiii}{$\bullet$}
\renewcommand{\labelitemiv}{$\bullet$}

\begin{itemize}
    \item Montag, 08. Oktober
        \begin{itemize}
            \item 10:00 Uhr: Gemeinsames Frühstück (Plaza, IZ 1.OG). \emph{Bitte eigenes Geschirr und Besteck mitbringen}
            \item 13:00 - 17:00 Uhr: Vorkurs
        \end{itemize}
    \item Dienstag, 09. Oktober
        \begin{itemize}
            \item 10:00 Uhr: Vorkurs
            \item 10:00 Uhr: Campustour (Fachgruppenraum Informatik IZ, 149/150)
            \item 13:00 - 17:00 Uhr: Vorkurs
            \item 19:00 Uhr: Kneipentour (Haupteingang Mensa 1)
        \end{itemize}
        %\newpage
    \item Mittwoch, 10. Oktober
        \begin{itemize}
            \item 10:00 - 16:00 Uhr: Vorkurs
            \item 16:00 Uhr: Linux-Install-Party (IZ 161)
        \end{itemize}
    \item Donnerstag, 11. Oktober
        \begin{itemize}
            \item 10:00 - 12:00 Uhr: Info-Vortrag \& Studenplanbau-Workshop (IZ 161)
            \item 13:00 - 17:00 Uhr: Vorkurs
            \item Abends: Erstsemesterparty der Fachschaften Fakultät 1 (Bienroder Weg 97, Raum -132, Keller)
        \end{itemize}
    \item Freitag, 12. Oktober
        \begin{itemize}
            \item 10:00 - 17:00 Uhr: Vorkurs
        \end{itemize}
    \item Montag, 15. Oktober
        \begin{itemize}
            \item 09:00 Uhr: Zentrale Begrüßung aller Erstsemester (Eintracht-Stadion)
            \item 10:30 - 12:00 Uhr: Infobörse \textquote{Studium ist mehr} (Foyer Altebäude/Audimax)
            \item 11:30 - 13:00 Uhr: Vorlesung \textquote{Lineare Algebra für Informatiker} (PK 2.2)
            \item 13:15 - 14:15 Uhr: Begrüßung durch das Department Informatik (PK 2.2)
            \item 15:00 - 16:30 Uhr: Vorlesung \textquote{Programmieren 1}
            \item Abends: Uniweiter Erstsemesterparty (Diskothek Jolly Time)
        \end{itemize}
    \item Dienstag, 16. Oktober
        \begin{itemize}
            \item 10:00 - 16:00 Uhr: Studium Generale (Verteilt über Hauptcampus)
            \item 19:00 Uhr: Analoger Spieleabend (Flur vor IZ 150)
        \end{itemize}
    \item Mittwoch, 17. Oktober
        \begin{itemize}
            \item 15:00 - 16:30 Uhr: Vorlesung \textquote{Lineare Algebra für Informatiker} (PK 2.2)
        \end{itemize}
    \item Donnerstag, 18. Oktober
        \begin{itemize}
            \item 08:00 - 09:30 Uhr: Übung \textquote{Programmieren 1} (PK 15.1)
        \end{itemize}
    \item Freitag, 19. Oktober
        \begin{itemize}
            \item 09:45 - 11:15 Uhr: Übung \textquote{Lineare Algebra für Informatiker} (PK 2.2)
            \item 14:00 Uhr: Treffen Abfahrt Erstifahrt (Foyer IZ, Mühlenpfordtstraße 23)
        \end{itemize}
    \item Ersti-Wochenende
        \begin{itemize}
            \item Wann? 19. - 21. Oktober
            \item Wo? Naturfreundehaus Eichsfelder Hütte (St. Andreasberg)
            \item Was? Lerne deine Mitstudierenden kennen, habe Spaß :)
            \item Finanzierung? Größtenteils aus Studienqualitätsmitteln, dazu 20 - 30 Euro Selbstkostenbeitrag
            \item Fristen: Anmeldung und Bezahlung des Selbstkostenbeitrags bis 16. Oktober
            \item Weitere Informationen findest du in unserem Info-Schreiben zur Ersti-Fahrt
            \item Die Anmeldung ist online möglich\footnote{\verUrl{5}{https://pretix.coldney.de/fsinfo/erstifahrtss2018/}}
        \end{itemize}
\end{itemize}
