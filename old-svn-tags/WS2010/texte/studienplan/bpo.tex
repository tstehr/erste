\label{po}\subsection{Die Prüfungsordnung}
An einer Universität gibt es tausende Regeln und Ordnungen. In den letzten Monaten war für dich die Zulassungsordnung das Maß aller Dinge, aber nun, wo du hoffentlich zum Studium zugelassen bist, ist die Prüfungsordnung das A und O und enthält Antworten auf 95\% aller Fragen, die im Studium auftreten - nicht nur wenn es um die eigentlichen Prüfungen geht. 

Unter der Annahme dass du für Xyz entweder Bachelor oder Master einsetzt, ist der konkrete Name "`Besonderer Teil der Prüfungsordnung für den Xyzstudiengang Informatik der Technischen Universität Braunschweig"', der oft als "`BPO Xyz"' "`XPO"' abgekürzt wird. Schnell merkst du, dass "`BPO"' sich somit auf den Bachelor beziehen kann oder auf beide Studiengänge, und ob es die Abkürzung "`MPO"' wirklich gibt, ist in gewissen Kreisen immernoch ein heiß debattiertes Thema.

Zudem gibt es noch eine APO, die Allgemeine Prüfungsordnung. Sie gilt Uniweit für alle Studiengänge, doch die beiden BPOs überschreiben die meisten APO-Regelungen sowieoso. Dennoch dachten wir, du solltest zumindest einmal von der APO gehört haben.

Zurück zur $BPO/MPO/BPO(B)+BPO(M)$\ldots{} Wie auch immer das Ding heißen mag: Es ist wichtig! Und da es weder besonders lang ist, noch kompliziert geschrieben ist, kann man eigentlich erwarten, dass jeder Student es mindestens einmal komplett liest. Wir würden sogar so weit gehen, dass man sie vor jedem Semesterbeginn nochmal lesen sollte, um die richtigen Entscheidungen für's nächste Semester zu treffen, denn sich den Inhalt über mehr als 6 Monate zu merken ist auch nicht ganz einfach. Und zu allem Überfluss ändert sich diese Ordnung fast jedes Semester, also ein Grund, ab und zu hineinzuschauen.

Noch eine wichtige Sache dazu: Überall wird man dich mit Ratschlägen, Regeln und Warnungen überhäufen. Nichts davon ist verbindlich, nicht dieses Heft und ebensowenig die mündlichen Aussagen und E-Mails von irgendwelchen Uni-Mitarbeiten. Nichtmal die offiziellen Beratungsstellen geben verbindliche Auskünfte! Alles was zählt, sind die Ordnungen, das Modulhandbuch (MHB, mehr dazu weiter unten) und gegebenenfalls noch Beschlüsse. Die folgenden Absätze sollen dir helfen, einen Überblick zu gewinnen, aber sie nehmen es dir nicht ab, in den Ordnungen die Details nachzulesen.

Wenn du es noch nicht getan hast, lade dir die für dich gültige Fassung am besten von \url{http://www.tu-braunschweig.de/fk1/service/informatik/dokumente} herunter. Vielleicht ist es ratsam, erst dieses Heftchen weiter zu lesen, und danach erst die MPO, oder gegebenenfalls beim Lesen dieses Textes hier und da mal ein Detail in der MPO ober im MHB nachzuschlagen.

%\end{multicols}
%\includegraphics[width=\textwidth]{bilder/comics/dilbert.png}
%\begin{multicols}{2}
