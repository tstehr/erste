\documentclass[
  final,
  a4paper,              % DIN A4
  style=screen,
  nexus,                % corporate design font
  bookmarkpackage=false,
]{tubsartcl}

\usepackage[utf8]{inputenc}
\usepackage{graphicx}
\usepackage{amssymb}
%\usepackage[T1]{fontenc}
\usepackage{multicol}
\usepackage{comment}
\usepackage[ngerman]{babel}
\usepackage{wrapfig}
\usepackage[babel,german=quotes]{csquotes}
\usepackage[obeyFinal]{todonotes}
\usepackage{etoolbox}
\usepackage{sectsty}
\usepackage[hidelinks]{hyperref}
\usepackage{tabularx}
\usepackage{pdfpages}
\usepackage{varwidth}
\usepackage{enumitem}
\usepackage{eurosym}
\usepackage{footnote}
\usepackage{booktabs}

\usetikzlibrary{patterns}

%\KOMAoptions{twoside, headsepline}
\usepackage{geometry}

\geometry{margin=1.5cm}
\geometry{twoside}
%\geometry{bindingoffset=7mm}
\geometry{includehead} 
\geometry{includefoot}
\geometry{footnotesep=2em}
\geometry{footskip=\footskip-2em}
\geometry{headsep=1em}
\geometry{head=5em}
\geometry{nofoot}

%layout-experimente
\setlength{\columnsep}{20pt}
\pagestyle{useheadings}



% Toggles between winter term and summer term 
\newtoggle{winter}

% This system is meant to make updating the Erste for the new semster simple. Every semster gets a new version number that is larger than the previous one (assigned in config.tex). 
% By using \tocheck, defined below, todos can be left in the source and disabled one by one by incresing the version number of the tocheck. Once all todos are adressed, the new version can be released. Later all todos can be enabled again by incrementing the version number.
\newcounter{version}

% Defines a conditional todo. 2 mandatory arguments:
% 1st param, Valid to version: This todo has been adressd as of the given version
% 2nd param, Todo description: What needs to be done to adress this todo
\newrobustcmd{\tocheck}[2]{
	\ifnumless{#1}{\value{version}}{
		\todo[inline]{#2}
	}{}
}

% Versioned url. 2 mandatory arguments:
% 1st param, Valid to version: This url is still valid as of version
% 2nd param, URL: The url
\newrobustcmd{\verUrl}[2]{\ifnumless{#1}{\value{version}}{\todo[inline]{Check \url{#2}}}{}\url{#2}}
\newrobustcmd{\verHref}[4][]{\ifnumless{#2}{\value{version}}{\todo[inline]{Check \url{#3}}}{}\href[#1]{#3}{\nolinkurl{#4}}}

\newrobustcmd{\fginfoUrl}[0]{\verUrl{7}{https://fginfo.tu-braunschweig.de}}

%%

\newrobustcmd{\xkcd}[2]{
	\begin{center}
		\includegraphics[#1]{bilder/XKCD/#2}
	\end{center}
}


% creates a blank page

\newcommand{\blankpage}{
	\newpage
	\thispagestyle{empty}
	\mbox{}
	\newpage
}


% creates a stundenplan
\newenvironment{stundenplan}[6]{%
	\newcommand{\wTag}{#1/6}%
	\newcommand{\hPlan}{#2}%
	\newcommand{\hAbendHeader}{.5}%
	\newcommand{\hAbend}{1.6}%
	\newcommand{\tStart}{zeit(#3,#4)}%
	\newcommand{\tEnde}{zeit(#5,#6)}%

	\pgfmathdeclarefunction{zeit}{2}{\pgfmathparse{##1 + ##2 / 60}}%
	\pgfmathdeclarefunction{tmpYZeit}{1}{\pgfmathparse{-\hPlan * (##1-\tStart) / (\tEnde-\tStart)}}%
	\pgfmathdeclarefunction{yZeit}{1}{\pgfmathparse{%
		max(min(tmpYZeit(##1), tmpYZeit(\tStart)), tmpYZeit(\tEnde)-\hAbendHeader-\hAbend)%
	}}%

%
	\tikzset{%
	  termin/.style={%
	   anchor=north west,%
	   align=left,%
	   %execute at begin node=\setlength{\baselineskip}{1.2em}%
	  }%
	}%

	\newcommand{\tNode}[2]{%
		\node [termin] at (TERMIN) {%
			\begin{varwidth}{1cm*\wTag - .2cm}%
			##1\\%
			\scriptsize ##2%
			\end{varwidth}%
		};%
	}%

%
	\newcommand{\Termin}[8]{%
		%1: Beschreibung, 2: Ort, 3: Tag, 4: Start Stunde, 5: Start Minute, 6: Ende Stunde, 7: Ende Minute, 8: Farbe %
		\draw [##8](\wTag * ##3,{yZeit(zeit(##4,##5))}) coordinate (TERMIN) rectangle (##3  * \wTag + \wTag,{yZeit(zeit(##6,##7))});%
		\tNode{##1}{##2}%
	}%

	\newcommand{\termin}[8]{%
		%1: Beschreibung, 2: Ort, 3: Tag, 4: Start Stunde, 5: Start Minute, 6: Ende Stunde, 7: Ende Minute, 8: Farbe (Fügt Zeit automatisch in Beschreibung ein)%
		\Termin{##1}{##4:##5 -- ##6:##7\ifstrempty{##2}{}{, ##2}}{##3}{##4}{##5}{##6}{##7}{##8}%
	}%

	\newcommand{\abendtermin}[4]{%
		%1: Beschreibung, 2: Ort, 3: Tag, 4: Farbe%
		\draw [##4](\wTag * ##3,-\hPlan-\hAbendHeader) coordinate (TERMIN) rectangle (##3  * \wTag + \wTag, -\hPlan-\hAbendHeader-\hAbend);%
		\tNode{##1}{##2}%
	}%
	\begin{tikzpicture}[font=\small]%
	% spalten
	\foreach \x in {0,...,5}{
		\draw (\x*\wTag,0) -- (\x*\wTag,-\hPlan);
		\draw (\x*\wTag,-\hPlan-\hAbendHeader) -- (\x*\wTag,-\hPlan-\hAbendHeader-\hAbend);
	}
	\draw (6*\wTag,0) -- (6*\wTag,-\hPlan-\hAbendHeader-\hAbend);

	\draw (0,0)--(6*\wTag,0);
	\draw (0, -\hPlan) coordinate (ABEND) rectangle (5*\wTag, -\hPlan-\hAbendHeader);
	\node [anchor=north west,align=center,minimum width=5cm*\wTag] at (ABEND) {\scriptsize \emph{Abend}};
	\draw (0, -\hPlan-\hAbendHeader-\hAbend)--(6*\wTag, -\hPlan-\hAbendHeader-\hAbend);

	\node [anchor=south] at (.5* \wTag,0) {\textbf{Montag}};
	\node [anchor=south] at (1.5* \wTag,0) {\textbf{Dienstag}};
	\node [anchor=south] at (2.5* \wTag,0) {\textbf{Mittwoch\vphantom{g}}};
	\node [anchor=south] at (3.5* \wTag,0) {\textbf{Donnerstag}};
	\node [anchor=south] at (4.5* \wTag,0) {\textbf{Freitag}};
	\node [anchor=south] at (5.5* \wTag,0) {\textbf{Wochenende\vphantom{g}}};
}{\end{tikzpicture}}

\renewcommand{\familydefault}{\sfdefault}

%\clubpenalty = 10000
%\widowpenalty = 10000 
%\displaywidowpenalty = 10000

% Trennregeln
\hyphenation{
	AStA
	Mit-be-wohnerIn-nen
	Pro-fessorInnen
	erwischt
	viiieel
	y-Nummer
	Uniaccount
	Andreas-berg
}

% left aligned sections
%\allsectionsfont{\raggedright}


\settoggle{winter}{false}
\setcounter{version}{7}



%\setlength{\parindent}{0cm}
\setlength{\parskip}{0em}
\geometry{head=0em}



\begin{document}

\pagestyle{empty}


\begin{center}
\noindent\includegraphics[width=.5\textwidth]{bilder/fg-logo/fg-logo.pdf}
\end{center}

\listoftodos


% <temp> (no erstifahrt mode)
\vskip5em
\begin{addmargin}[6em]{6em}
% </temp>

%\begin{multicols}{2}

\noindent{}Wir, der Fachgruppenrat Informatik, heißen dich herzlich an der TU Braunschweig willkommen. 
Wir haben dir viel zu sagen – viel mehr, als hier auf eine Seite passt. Daher solltest du unbedingt auch folgende Infos zur Kenntnis nehmen:

\begin{itemize}[topsep=0em,itemsep=-1ex,partopsep=0em,parsep=1ex]
\item Unser Blog, in dem du aktuelle Ankündigungen findest (und ggf. Änderungen), sowie Kontakte zu Ansprechpartnern: \\ \fginfoUrl

\item cs-studs -- die Mailingliste, über die alle wichtigen Infos bekannt gegeben werden. Am besten jetzt sofort eintragen: \\\verUrl{8}{https://tinyurl.com/csstuds}

\item Master-Studierende, die Zulassungsauflagen erhalten haben, sollten außerdem die FAQ\footnote{\verUrl{8}{https://wiki.fginfo.tu-bs.de/doku.php?id=infos:faq}} lesen.
\end{itemize}

\noindent{}Wir sind ebenfalls Studierende der Informatik, und arbeiten als deine Vertretung in diesem Studiengang. Das bedeutet, dass wir dich in verschiedenen Gremien der Universität vertreten und somit aktiv an der ständigen Verbesserung des Informatikstudiums arbeiten. Um selbst Ideen/Vorschläge für die Verbesserung eures Studiums einzubringen oder wenn du Fragen zum oder Probleme im Studium hast, bist du zu den regelmäßigen Treffen der Fachgruppe herzlich eingeladen. Diese finden einmal wöchentlich im Raum 150 des Informatikzentrums statt – der konkrete Termin steht auf dem Blog.

Wir vertreten also deine Interessen, aber das ist bei weitem noch nicht alles! Zusätzlich sorgen wir auch noch für eine Menge Spaß neben dem Studium mit Veranstaltungen wie Grillen, Spieleabende, Glühweinabende und vieles mehr.

Nun freuen wir uns aber darauf, dich kennen zu lernen! Wir haben für die ersten Wochen einige Veranstaltungen organisiert, um dir die Orientierung an der Uni zu erleichtern. Auf der nächsten Seite findest du einen Überblick über die Orientierungsveranstaltungen, die von uns oder von Anderen angeboten werden.\footnote{Einzelne Termine können sich noch ändern. Unter \verUrl{8}{https://fginfo.tu-braunschweig.de/ersti/} findest du den aktuellen Plan.}

%\columnbreak

%\section*{Erstifahrt}

%Auch in diesem Semester werden wir eine Ersti-Fahrt veranstalten. Dabei handelt es sich um ein Wochenende, das wir gemeinsam außerhalb des Uni-Campus verbringen werden. 

%Die Fahrt wird aus Studienqualitätsmitteln finanziert. Wir bitten lediglich um eine Selbstbeteiligung in Höhe von \tocheck{8}{Teilnehmerbeitrag?}20 – 30 \euro{} (genauer Betrag wird noch bekannt gegeben). Hier nun die wichtigsten Informationen zusammengefasst:

%\textbf{Wann?} Die Fahrt wird vom \tocheck{8}{Datum}25. bis zum 27. Oktober 2019 stattfinden. Dabei werden wir am Freitagnachmittag vom Informatikzentrum in Braunschweig aus abfahren. Am Sonntag fahren wir gegen Mittag zurück. Genaue Informationen über Abfahrtszeiten und Treffpunkt werden wir noch rechtzeitig per E-Mail bekannt geben.

%\textbf{Wo?} Es geht in die Eichsfelder Hütte in St. Andreasberg im Harz.

%\textbf{Was?} Im Vordergrund dieser Veranstaltung steht, dass ihr die Möglichkeit bekommt euch als Studienanfänger kennen zu lernen. So möchten wir dir den Einstieg in das Studium erleichtern. Dabei soll der Spaß natürlich nicht zu kurz kommen! Es ist ein umfangreiches Programm geplant. Außerdem wirst du eventuell offene Fragen über das Studium loswerden können.

%Die Platzzahl für diese Fahrt ist begrenzt, also melde dich so früh wie möglich an. Die Anmeldung erfolgt elektronisch über ein von uns eingerichtetes Anmeldeformular, das du unter \verUrl{8}{https://fginfo.tu-braunschweig.de/ersti/} findest. Beachtet dabei bitte, dass wir die Anmeldung und den Selbstkostenbeitrag bis spätestens zum \tocheck{8}{Letzter Anmeldetermin}23.10 brauchen. Das Geld kannst du dazu einfach zu einer unserer Veranstaltungen mitbringen.

%\end{multicols}

% <temp> (no erstifahrt mode)
\end{addmargin}
% </temp>

\vskip5em

\begin{center}
\noindent{}Viele Grüße und einen erfolgreichen Start ins Studium \\
Wünscht der Fachgruppenrat Informatik
\end{center}

\clearpage
\input{texte/stundenplan/style_sw}
\section*{Einführungs- und Orientierungsveranstaltung zum Semesterbeginn}

\tocheck{8}{Stundenpläne überarbeiten}


\begin{savenotes}

\subsubsection*{6. April -- 10. April}

\begin{centering}
\begin{stundenplan}{18}{5.5}{9}{00}{17}{00} % width, height, starth, startmin, endh, endmin
	\Termin{Ersti-Frühstück\footnote[1]{Bitte eigenes Geschirr, Besteck und Tasse mitbringen}}
		{10:00, Plaza (IZ\footnote[2]{
			IZ:~Informatik-Zentrum,~Mühlenpfordstr.~23~|
			Mensa~1:~Katharinenstraße~1~| 
			%Eintracht-Stadium:~Hamburger~Str.~21~|
			PK:~Pockelstr.~| 
			SN:~Schleinitzstr.~| 
			Tentomax:~Konstantin-Uhde-Straße~| 
			%BI:~Campus~Nord,~Bienroder~Weg~97~| 
			Campusplan:~\verUrl{9}{https://campusplan.tu-braunschweig.de}
			\label{rooms}}~1.~OG)
		}{0}{10}{00}{11}{15}{fgTermin}
	\Termin{Stundenplanbauen}{11:00, IZ~160\footref{rooms}}{0}{11}{15}{13}{00}{fgTermin}
	\termin{Informatik-Vorkurs}{}{0}{13}{00}{17}{00}{uniTermin}

	\termin{Informatik-Vorkurs}{}{1}{9}{00}{13}{00}{uniTermin}
	\Termin{Mensa-Besuch}{13:00 -- 14:00, Mensa 1\footref{rooms}}{1}{13}{00}{14}{15}{fgTermin}
	\Termin{Campustour}{14:00, Treffen Foyer IZ\footref{rooms}}{1}{14}{15}{17}{00}{fgTermin}

	\termin{Informatik-Vorkurs}{}{2}{9}{00}{13}{00}{uniTermin}
	\Termin{Nachmittags-programm}{14:00, du kannst zwischen mehreren Optionen auswählen}{2}{14}{00}{17}{00}{fgTermin}
	\abendtermin{Linux-Install-Party}{17:00, IZ~160\footref{rooms}}{2}{fgTermin}

	\termin{Informatik-Vorkurs}{}{3}{9}{00}{13}{00}{uniTermin}
	\Termin{Nachmittags-programm}{14:00, du kannst zwischen mehreren Optionen auswählen}{3}{14}{00}{17}{00}{fgTermin}
	\abendtermin{Kneipentour}{19:00, Start Haupteingang der Mensa 1\footref{rooms}}{3}{fgTermin}

	%\termin{Informatik-Vorkurs}{}{4}{11}{00}{16}{00}{uniTermin} %Feiertag im Sommer2020	
\end{stundenplan}
\end{centering}

\vskip-100ex

\subsubsection*{13. April -- 17. April}

\begin{centering}
\begin{stundenplan}{18}{9}{8}{00}{16}{00} % width, height, starth, startmin, endh, endmin
	\termin{Analysis (V)}{PK~2.2\footref{rooms}}{0}{11}{30}{13}{00}{vlTermin}
	\Termin{Erstibegrüßung der Informatik}{13:15 -- 14:15, PK~2.2\footref{rooms}}{0}{13}{15}{14}{30}{uniTermin}

	\termin{Algebra (V)}{PK~2.2\footref{rooms}}{1}{9}{45}{11}{15}{vlTermin}
	\termin{Analysis (V)}{PK~2.2\footref{rooms}}{1}{11}{30}{13}{00}{vlTermin}

	\termin{Programmieren (V)}{Tentomax\footref{rooms}}{2}{8}{00}{9}{30}{vlTermin}
	\termin{Logik (V)}{PK 2.2\footref{rooms}, \emph{Letzter Termin für die Anmeldung zur Erstifahrt}}{2}{9}{45}{11}{15}{vlTermin}
	\termin{Programmieren (Ü)}{SN 19.1\footref{rooms}}{2}{11}{30}{13}{00}{vlTermin}
	\Termin{Infobörse für Erstsemester}{11:00 -- 16:00, Wiese vor der Mensa}{2}{13}{00}{16}{00}{uniTermin}

	\abendtermin{Spieleabend}{19:00, Flur vor IZ~150\footref{rooms}}{2}{fgTermin}

	\Termin{Treffen der Medizininformatiker}{
		15:00, IZ 404\footref{rooms}, \emph{
			Nur für Medizininformatikstudierende\footnote[3]{Um Anmeldung bei wird gebeten. Wenn du Medizininformatik studierest und teilnehmen willst, schicke eine Mail an ute.zeisberg@plri.de}
		}
	}{3}{14}{15}{16}{00}{uniTermin}


	\termin{Algebra (Ü)}{PK 2.2\footref{rooms}}{4}{8}{00}{9}{30}{vlTermin}
	\termin{Analysis (Ü)}{PK 2.2\footref{rooms}}{4}{9}{45}{11}{15}{vlTermin}
	\Termin{Abfahrt Erstifahrt}{Foyer IZ\footref{rooms}}{4}{13}{30}{14}{30}{fgTermin}

	\Termin{Erstifahrt}{Naturfreundehaus St. Andreasberg im Harz, Abreise Sonntag ab Mittag}{5}{0}{0}{24}{0}{fgTermin}

	% \Termin{Zentrale Erstibegrüßung}{9:00, Eintracht-Stadium}{0}{8}{45}{10}{00}{uniTermin}
	% \Termin{Infobörse}{10:30 -- 12:00, Altgebäude \& Audimax}{0}{10}{15}{11}{30}{uniTermin}
	% \termin{Lineare Algebra (V)}{PK~2.2\footref{rooms}}{0}{11}{30}{13}{00}{vlTermin}
	% \Termin{Erstibegrüßung der Informatik}{13:15 -- 14:15, PK~2.2\footref{rooms}}{0}{13}{15}{14}{30}{uniTermin}
	% \termin{Programmieren (V)}{PK~15.1\footref{rooms}}{0}{15}{00}{16}{30}{vlTermin}
	% \abendtermin{Uniweite Erstsemesterparty}{Diskothek Jolly Time}{0}{uniTermin}

	% \termin{Studium Generale}{Verteilt über Hauptcampus}{1}{10}{00}{16}{00}{uniTermin}
	% \abendtermin{Spieleabend}{19:00, Flur vor IZ~150\footref{rooms}, \emph{Letzter Termin für die Anmeldung zur Erstifahrt}}{1}{fgTermin}

	% \termin{Diskrete Mathematik (V)}{PK~2.2\footref{rooms}}{2}{9}{45}{11}{15}{vlTermin}
	% \termin{Lineare Algebra (V)}{PK~2.2\footref{rooms}}{2}{15}{00}{16}{30}{vlTermin}

	% \termin{Programmieren (Ü)}{PK~15.1\footref{rooms}}{3}{8}{00}{9}{30}{vlTermin}

	% \termin{Lineare Algebra (Ü)}{PK~2.2\footref{rooms}}{4}{9}{45}{11}{15}{vlTermin}
	% \Termin{Abfahrt Erstifahrt}{14:00, Foyer IZ\footref{rooms}}{4}{13}{30}{14}{30}{fgTermin}

	% \Termin{Erstifahrt}{Naturfreundehaus St. Andreasberg im Harz, Abreise Sonntag ab Mittag}{5}{0}{0}{24}{0}{fgTermin}
\end{stundenplan}

\end{centering}

%\enlargethispage{\baselineskip}
\end{savenotes}


\end{document}
