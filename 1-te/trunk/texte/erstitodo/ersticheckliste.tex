\begin{figure}[h]

\begin{tabular}{|c|c|c|c|c|}
\hline &\textbf{Bis} 						& \textbf{Todo} 				& \textbf{Seite}& \textbf{Muss?} \\ 
\hline & Ende Oktober 						& Bafög beantragen 				&  				& nein \\ 
\hline & Vor Studienbeginn 					& GITZ-Account freischalten 	&  				& ja \\ 
\hline & 									& Mailinglisten 				&  				& nein \\ 
\hline & Beim ersten Mensa-Besuch 			& Mensa-Card 					&  				& nein \\ 
\hline & 									& Bibliotheksausweis 			&  				& nein \\
\hline & 1 Woche nach Umzug 				& Wohnsitz Ummelden 			&  				& ja \\ 
\hline & Anmeldewoche (Dezember) 			& Prüfungsanmeldung 			&  				& ja \\ 
\hline & Bevor man an der Uni online geht 	& WLAN einrichten 				&  				& nein \\ 
\hline & spätestens Dezember 				& Prüfungsbogen 				&  				& ja \\ 
\hline & 									& Auflagen klären 				&  				& Master \\ 
\hline & 									& Kopierkarte 					&  				& nein \\ 
\hline & 									& Blog abonnieren 				&  				& sollte :) \\ 
\end{tabular} 
\end{figure}


\subsection{Ersti Checkliste}

Hier wird zusammengefasst, was ihr in den ersten Tagen des Studiums
unbedingt erledigen solltet. Ihr könnt die Punkte in der folgenden
Tabelle abhaken um den Überblick zu behalten.

Im Folgenden findet ihr noch erweiterte Infos zu manchen der Punkte. 
Andere Hinweise sind durch die restliche Zeitung verteilt, dafür 
gibts eine Spalte mit der jeweiligen Seite.


\subsubsection{BAf"oG}

Wer BAf"oG beantragen m"ochte, sollte sich am besten gr"undlich
informieren. Sehr zu empfehlen ist da: \\
\mbox{\nurl{http://www.bafoeg.bmbf.de/}}
 
F"orderungsantr"age gibt es zum Download oder in Papierform im
EG des BAf"oG-Amtes, Nordstra"se 11. Am besten so fr"uh
wie m"oglich beantragen, denn BAf"oG wird nicht r"uckwirkend
bezahlt.

\newpage

\vspace*{9.5cm}

\subsection{GITZ-Account}
Unser Rechenzentrum, das Gauß-IT-Zentrum, stellt euch diverse 
Dienste zur Vefügung, wovon manche quasi lebenswichtig sind, 
andere eher nebensächlich. Aber für all diese Dienste braucht 
ihr eine GITZ-Account-Nummer und ein Passwort. Diese so genannte 
y-Nummer ist nicht das gleiche wie eure Immatrikulationsnummer. 
In der Regel bekommt ihr schon vor Semesterbeginn eine Nummer 
und ein vorläufiges Passwort per Post zugesendet. Dieses 
Passwort müsst ihr euch nicht mehrken, denn ihr braucht es nur 
einmal, nämlich um sich ein richtiges Passwort für die spätere 
Verwendung auszusuchen. Das solltet ihr auf jeden Fall möglichst 
früh von einem eigenen PC von zuhause aus machen (denn ohne das 
gemacht zu haben, stehen euch die Uni-PCs nicht zur Verfügung, 
und ihr kommt in der Uni auch noch nicht ins WLAN). Dann solltet 
ihr euch alle drei wichtigen Daten - Matrikelnummer, Y-Nummer 
und das neue Passwort gut einprägen (ihr braucht sie dann 
ständig zu den unmöglichsten Zeiten), gegebenenfalls auch 
aufschreiben und sicher verwahren.

Es kann auch passieren, dass ihr den besagten Brief vom GITZ 
gar nicht bekommt, dann müsst ihr euch selbst um all das kümmern. 
Keine Sorge, das passiert halt ab und zu, ist aber nicht weiter 
schlimm.

\subsubsection{Mailingliste und Emailadresse}

Es gibt eine Mailingliste f"ur die Studierenden der Informatik.
Sie hei"st \emph{cs-studs} und ist \emph{die} Informationsquelle.
Hier werden Ank"undigungen zu Lehrveranstaltungen gemacht, eure
Fachgruppe k"undigt hier Spiele- und Grillabende an und es gibt
oft Angebote zu Hiwistellen oder offenen Teamprojekten,
Bachelorarbeiten etc. und selbstverst"andlich ist dies auch ein
guter Ort, um Fragen zum Studium loszuwerden.

Anmelden k"onnt ihr euch unter
\url{http://www.cs.tu-bs.de/mailinglisten.html}.

Zusammen mit eurem GITZ-Account bekommt ihr auch ein neues 
Email-Postfach mit bis zu drei Adressen (y00000000@tu-bs.de, 
vorname.nachname@tu-bs.de, v.nachname@tu-bs.de). Für die oben 
genannte Mailingliste, und diverse andere Zwecke, könnt ihr euch 
meist aussuchen, ob ihr eure vorherige private Emailadresse nutzt, 
oder die neue von der TU-Braunschweig. Aber egal wie ihr euch 
entscheidet, ab und zu erreichen euch auch Emails auf eurem 
TU-Braunschweig-Postfach, also schaut dort regelmäßig rein!

\subsubsection{IRC-Channel und Forum/Wiki}

Viele Studierenden der Informatik, Nebenfachh"orer und
Fachgruppenmitglieder sind im IRC-Channel \texttt{\#\#cs-studs}
(ja, der zweite ,,\#'' ist korrekt) auf \texttt{irc.freenode.net}
unterwegs. Auch hier ist ein guter Ort, Fragen zu stellen.

Unter \url{http://clevershit.de} findet ihr außerdem ein Forum und ein 
Wiki extra für Informatiker an der TU-Braunschweig, auf dem ihr Fragen 
stellen könnt und extrem viele nützliche Infos für's Studium findet. Um 
euch dort anzumelden, braucht ihr übrigens die TU-Braunschweig-Emailadresse, 
die ihr vom GITZ bekommt.

\subsubsection{Mensa-Card}

Ihr braucht unbedingt eine Mensa-Card (eine Chipkarte,
mit der ihr in der Mensa bargeldlos bezahlen k"onnt), sonst
m"usst ihr den G"astepreis zahlen. Bei der Immatrikulation bekommt ihr 
einen Gutschein, den ihr gegen die Mensacard eintauschen könnt - falls 
nicht, kann man sie auch einfach für 5 Euro erwerben. Ihr solltet auch 
Studierendenausweis und Lichtbildausweis nicht vergessen, auch wenn 
das nicht immer gewissenhaft kontrolliert wird. Sobald ihr die Karte 
habt, schreibt euch die darauf stehende Nummer auf, so könnt ihr eurer 
Restgeld wiederbekommen, falls ihr die Karte einmal verliert - und dass 
passiert einem leider recht oft.

\subsubsection{Uni-Bibliothek}

Um B"ucher in der Uni-Bibliothek ausleihen zu k"onnen, braucht ihr
einen Ausweis. Diesen k"onnt ihr an einem der Terminals in der
Bibliothek beantragen und danach gegen eine Geb"uhr von \EUR{5}
am Schalter abholen. Je nachdem, ob ihr zu Beginn schon Bücher braucht, 
könnt ihr die Karte auch einfach ein bisschen später besorgen.

In der Bibliothek stehen außerdem Kopierer bereit, die ihr nutzen könnt. 
Einen davon könnt ihr mit Kleingeld befüllen, kompfortabler geht es aber 
mit einer Kopierkarte. Und auch diese bekommt ihr für ein paar Euro 
direkt in der Bibliothek.

Zu Semesterbeginn gibt es oft noch Einführungskurse in die 
Bibliotheksbenutzung. Ob ihr eure Bibliothekskarte vor oder nach 
diesem Kurz besorgt, ist egal.

\subsubsection{Ummelden}

Wer neu nach Braunschweig gezogen ist, muss sich innerhalb einer
Woche beim Einwohnermeldeamt anmelden. Wenn ihr die Frist verpasst, 
drohen theoretisch Strafen, aber praktisch sieht es da nicht so 
streng aus. Wenn man Braunschweig als
Erstwohnsitz w"ahlt, bekommt man (ein Jahr später) eine einmalige 
Zuzugspr"amie von
\EUR{200} (Immatrikulationsbescheinigung nicht vergessen). Wer
dennoch seinen Erstwohnsitz in der Heimat behalten m"ochte, sollte
glaubhaft darlegen k"onnen, dass er mehr als die H"alfte des Jahres
nicht in Braunschweig lebt bzw. seinen Lebensschwerpunkt in der
Heimatstadt hat.

\subsubsection{Pr"ufungsanmeldung}

Ihr m"usst euch f"ur alle Pr"ufungen, an denen ihr teilnehmen
wollt, vorher beim Pr"ufungsamt anmelden. \emph{Das ist nur eine
Woche lang m"oglich}, im Wintersemester meistens Mitte Dezember,
\emph{informiert euch also rechtzeitig, wann genau das ist}!

Vor eurer ersten Pr"ufungsanmeldung m"usst ihr au"serdem ein
Datenblatt ausf"ullen. Es empfiehlt sich, das bereits vor der
Anmeldewoche zu machen, weil die Schlangen dann nicht so lang sind.

Dar"uber hinaus gibt es die M"oglichkeit, sich online f"ur
Pr"ufungen anzumelden. Dazu braucht ihr allerdings eine TAN-Liste,
die ihr euch vorher im Pr"ufungsamt organisieren m"usst.

Weil das ganze etwas komplizierter ist, und euch in den allerersten 
Wochen noch nicht so direkt tangiert, machen wir da zu gegebener Zeit 
noch einen extra Infotermin, der euch alles wichtige dazu erklärt.