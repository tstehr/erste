\section{Nach dem Bachelor zur TU-Braunschweig}

Noch vor kurzem speiste sich der Informatik-Master der TU-Braunschweig fast ausschließlich durch die ansässigen Bachelor-Studenten, doch erfreulicherweise hat sich der Master nun auch anderswo herum gesprochen. Wer seinen Bachelor woanders erworben hat, steht im ersten Mastersemester vielen kleinen und mittelgroßen Schwierigkeiten gegenüber, denn die meisten Einfühungsveranstaltungen und -texte richten sich an Bachelor-Erstis, und längst nicht alles davon trifft auch auf neue Masterstudenten zu. Ich habe vor zwei Semestern selbst diesen Umstieg bzw. Umzug gewagt und versuche mal zu rekapitulieren, was mir leider damals keiner sagte\ldots

\subsection{Herden, Rudel und Einzelgänger}
Der recht feste Stundenplan im Bachelor-Studium sorgt dafür, dass man dort in der Regel mit vielen Kommilitonen zusammensitzt, die in der gleichen Situation sind wie man selbst: Neu hier und mit den gleichen Fragen und Sorgen. Und ist ein Block zu Ende, so zieht man gemeinsam zum nächsten Raum, wo man mit praktisch der gleichen Gruppe das nächste Fach abgrast. Eine typische Herde also.

Im Master ist das Grundlegend anders. Jeder hört andere Vorlesungen, und in den oft so genannten "`Mastervorlesungen"' tummeln sich Bachelor-, Master- und Diplomstudenten aus diversen Jahrgängen, oft auch aus anderen artverwandten Studiengängen wie z.B. Wirtschaftsinformatik. Da kann es eine ganze Weile dauern, bis man gecheckt hat, wer nun auch im Masterstudium ist, geg. Falls noch im gleichen Jahrgang, und selbst dann haben diese Leute ihren Bachelor hier oder dort, in diesem oder jenem Fach an einer Uni oder FH gemacht. Vielleicht haben die neben dir zuvor ganz andere Dinge gelernt, vielleicht sind sie hier um sich auf etwas komplett anderes zu spezialisieren als du.

Keine Frage: Diese Mischung macht es spannender, bunter und vielseitiger. Aber auf jeden Fall auch schwieriger. Wir können hier kaum Tipps geben, wie man als Neuling und eventuell unfreiwilliger Einzelgänger ein kleines Rudel findet oder bildet (denn wenn man dem Informatiker-Clichee glaubt, wissen wir das nämlich selbst nicht\ldots). Weder wir noch dieses Heft könnten all das ersetzen, was eine Gruppe von Gleichgesinnten mit gleichen Problemen und Interessen könnte. Aber wir wissen, dass man in den ersten Tagen und Wochen viele Fragen hat, und gerade als Master oft nur wenige an der Seite, der die gleichen Fragen und/oder passende Antworten haben. Deshalb wollen wir euch alle wichtigen Infos mitgeben, die man nicht unbedingt rechtzeitg per Hörensagen mitbekommen würde.

Und nicht dass du jetzt denkst, du würdest als Einzelgänger bis ans Ende deiner Studenlaufbaun verlassen und allein dahin vegitieren. Das wird schon noch, die Sache mit dem Rudel. Erwarte nur nicht zu viel von den ersten paar Tagen. Und gerade als Master-Student solltest du die vielfältigen Angebote der Fachgruppe (Spieleabende, Kneipentouren, Grillen, etc.) nutzen um die anderen kennen zu lernen - siehe \url{http://fginfo.cs.tu-bs.de/}

Nun also zu all dem, was du so schnell wie möglich wissen solltest.

\subsection{Die Prüfungsordnung}
An einer Universität gibt es tausende Regeln und Ordnungen. In den letzten Monaten war für dich die Zulassungsordnung das Maß aller Dinge, aber nun, wo du hoffentlich zum Masterstudium zugelassen bist, ist die Prüfungsordnung das A und O und enthält Antworten auf 95\% aller Fragen, die im Studium auftreten - nicht nur wenn es um die eigentlichen Prüfungen geht. Konkret gilt für dich der "`Besondere Teil der Prüfungsordnung für den Masterstudiengang Informatik der Technischen Universität Braunschweig"', der oft als "`BPO Master"' abgekürzt wird oder als "`MPO"' (ob es diese Abkürzung wirklich gibt, ist in gewissen Kreisen immernoch ein heißt debatiertes Thema). Die Abkürzung "`BPO"' allein kann sich je nach Kontext auf die Ordnung für Bachelor beziehen oder aber übergreifend für Bachelor und Master. Zudem gibt es noch eine APO, die Allgemeine Prüfungsordnung. Sie gilt Uniweit für alle Studiengänge, doch die beiden BPO's überschreiben die meisten APO-Regelungen sowieoso. Dennoch dachten wir, du solltest zumindest einmal von der APO gehört haben.

Zurück zur BPO/MPO/BPO(M)\ldots Wie auch immer das Ding heißen mag: Es ist wichtig! Und da es weder besonders lang ist, noch kompliziert geschrieben ist, kann man eigentlich erwarten dass jeder Student es mindestens einmal komplett liest. Ich würde sogar so weit gehen, dass man sie vor jedem Semesterbeginn nochmal lesen sollte, um die richtigen Entscheidungen fürs nächste Semester zu treffen, denn sich den Inhalt über mehr als 6 Monate zu merken ist auch nicht ganz einfach. Und zu allem Überfluss ändert sich diese Ordnung fast jedes Semester, also ein Grund, ab und zu hineinzuschauen.

Noch eine wichtige Sache dazu: Überall wird man dich mit Ratschlägen, Regeln und Warnungen überhäufen. Nichts davon ist verbindlich, nicht dieses Heft und ebensowenig die mündlichen Aussagen und Emails von irgendwelchen Uni-Mitarbeiten. Nichtmals die offiziellen Beratungsstellen geben verbindliche Auskünfte! Alles was zählt, sind die Ordnungen, das Modulhandbuch (MHB, mehr dazu weiter unten) und geg. Falls noch Beschlüsse. Die folgenden Absätze sollen dir helfen, einen Überblick zu gewinnen, aber sie nehmen es dir nicht ab, in den Ordnungen die Details nachzulesen.

Wenn du es noch nicht getan hast, lade dir die für dich gültige Fassung am besten von \url{http://www.tu-braunschweig.de/fk1/service/informatik/dokumente} herunter. Vielleicht ist es ratsam, erst dieses Heftchen weiter zu lesen, und danach erst die MPO, oder geg. Falls beim Lesen dieses Textes hier und da mal ein Detail in der MPO ober im MHB nachzuschlagen.

\subsection{Unterschiede zwischen den Bachelor-Abschlüssen}
Eventuell hat dein bisheriger Abschluss dir mehr als 180 Credit Points eingebracht - genau so viele hättest du nämlich in einem Bachelor an dieser TU erreicht. Theoretisch könnte es möglich sein, solche überschüssugen CPs auf den Master anzurechenen. Es ist uns nicht bekannt, ob das jemals jemand versucht hat, geschweige denn, ob es geklappt hätte, aber Fragen kostet ja zum Glück (auch hier) nichts.

Selbst bei gleicher Anzahl an CP ist der Bachelor an jeder Hochschule ein wenig anders, wobei Hochschule jetzt mal als Oberbegriff für Universität, Fachhochschule, Berufsakademie und viele andere Formen stehen soll. Zwischen den Universitäten in Deutschland herrscht eine formale Übereinkunft in den Inhalten, die in einem Bachelor-Studium namens "`Informatik"' vorkommen, daher wird in dem Fall von dir inhaltlich vermutlich nichts bedeutendes verlangt, was nicht auch an deiner Universität gelehrt wurde.

Falls du von einer Nicht-Universität (also z.B. Fachhochschule) und/oder aus einem Studiengang der sich nicht exakt "`Informatik"' nennt kommst und/oder dein Abschluss kein Bachelor of Science ist, dann kann es durchaus sein, dass du mit einer anderen Vorbildung hier her kommst als sie hier erwartet wird. Manche dieser Unterschiede sind schlichtweg egal, andere musst du selbst erkennen und ausgleichen, und bei gewissen Unterschieden "`wirst du geholfen"' diese zu beheben\ldots

\subsection{Zulassungsauflagen}
Ob man für das Masterstudium zugelassen ist, lässt sich leider nicht komplett durch true und false ausdrücken, denn das Immatrikulationsamt belegt manche von euch mit so genannten Zulassungsauflagen. Ob du eine solche Auflagen bekommst, steht in einem der ersten Briefe, die du von der TU erhältst - und in der Regel nur dort, also heb diesen Brief gut auf! Wenn du keine solchen Zulassungsauflagen hast, kannst du den restlichen Abschnitt gerne überspringen.

Es handelt sich dabei um Fächer aus dem Informatik-Bachelor, die du zusätzlich zu den Master-Fächern noch belegen musst - für die Note und die zu erreichenden Credit Points im Master zählen sie nicht. Wenn diese innerhalb des ersten Jahres nicht erbracht werden, dann war es das (theoretisch) mit dem Master, und selbst wenn du die Auflagenfächer bestehtst, aber dann nicht selbst dafür sorgst, diese Information vom Prüfungsamt zum Immatrikulationsamt zu tragen, droht nach dem zweiten Semester die Exmatrikulation. Lass dir diese Worte eine Warnung sein, aber sei beruhigt: wer durch ein Auflagenfach durchfällt oder den Nachweis vergisst, kann immernoch gewisse Schritte ergreifen, um weiter zu studieren - besser ist es aber, es nicht darauf ankommen zu lassen.

Der (eigentliche) Sinn hinter den Auflagen ist es, Defizite (im Vergleich zum TU-BS-Bachelor) auszugleichen, die du aus deinem vorherigen Studium mitbringst, d.h. Inhalte nachzuholen, die in deiner bisherigen Ausbildung zu kurz kamen oder ganz fehlten, und die für den erfolgreichen Abschluss des Masterstudiums nötig sind. Da nicht jeder die Diziplin hat, diese freiwillig aufzuarbeiten, versucht man dir hier durch den Zwang zu "`helfen"'.

Was vielleicht noch ganz nett klingt, ist im letzten Jahr ziemlich entartet, so dass jeder, der von der TU-Informatik-Bachelor-Norm abwich pauschal "`Theoretische Informatik 1"' und "`Einführung in die Logik"' nachholen musste - da war es dann auch egal, ob man die Fächer schon im Bachelor hatte und mit 1,0 abgeschlossen hatte. Ein Student hatte dagegen geklagt, und in Folge hatte er, und wenig später auch viele andere, die Auflage erlassen bekommen - einige wenige mussten sie trotzdem ableisten. Das einen solche Pflichten davon abhalten, die Inhalte nachzuholen, die einem wirklich fehlen, steht natürlich auf einem anderen Blatt.

Wir kennen das (hoffentlich verbessete) Auflagen-Vergabeverfahren für deinen Jahrgang nicht, aber hoffen natürlich, dass nun generell weniger eine Zulassungsauflage erhalten und wenn doch, dass sie dann wirklich die Inhalte betrifft, bei denen Defizite bestehen und die auch für den Master relevant sind. Bisher hieß solch eine Auflage, dass man die Vorlesung besuchen muss, die Hausaufgaben lösen und einreichen muss, und an der Klausur teilnehmen muss, die man auch unbedingt im ersten Versuch bestehen sollten. 

Für deinen Jahrgang ist nun einiges angenehmer geworden: Es soll nun erstmals möglich sein, zu Semesterbeginn freiwillig an einer mündlichen Prüfung teilzunehmen. Wird diese bestanden, dann ist die Auflage erfüllt, falls nicht, muss wie gehabt die Klausur belegt werden. Auch wird in den meisten Fächern die Hausaufgabe nicht mehr verpflichtend sein um an der Klausur teilzunehmen.

Falls du also eine Auflage erhalten hast, die dir fragwürdig erscheint, oder wenn man vergessen hat, dich über die möglichkeit der freiwilligen mündlichen Prüfung zu informieren, oder du sonst irgendwelche Fragen dazu hast, wende dich am besten an die Fachgruppe. Ratsam ist es auch, mit den anderen Erstis in deinem Jahrgang zu sprechen und zu vergleichen, wie deren Auflagen aussehen.

\subsection{Niveau ist keine Hautcreme\ldots}
\ldots aber auch nichts, was man direkt messen oder vergleichen kann. Die TU-Braunschweig hat eine recht hohe Meinung von ihrem Niveau - aber mal ehrlich, welche Hochschule würde auch etwas anderes von sich behaupten? In der Tat, es gibt hier hochqualitätive Lehre, engagierte Professoren und gewisse Mindestanforderungen an die Studierenden. Aber letzlich kochen auch hier alle mit Wasser, und man braucht als zugezogener Masterstudent keine übermäßige Angst vor dem Niveau-Unterschied zu haben. Der Sprung von der Schule zum Bachelorstudium war sicher größer - und den hast du ja offensichtlich geschafft, wenn du nun hier zum Masterstudium antrittst. Selbst wenn man "`nur"' von einer FH kommt - und die Vorbehalte bezüglich Fachhochschulen sind leider bei manchen Professoren groß - muss man nicht automatisch einen Einbruch im Notenschnitt befürchten.

Bemerkenswert ist, dass in vielen Master-Vorlesungen das Niveau mit fortschreitender Semesterzeit gegen unendlich strebt. Wenn nach 80\% des Semesters nur noch 5\% der Studenten verstehen, was gerade erklärt wird, ist das kein Grund zur Sorge - auch wenn du nicht zu diesen 5\% gehörst. Oft ist es so, dass man mit den ersten zwei Dritteln des Vorlesungsstoffes eine Note im 1er-Bereich bekommen kann - diese zwei Drittel sollte man dann natürlich möglichst perfekt beherrschen, und das ist auch nicht gerade einfach, aber machbar. Über kleinere Aussetzer und Fehler helfen praktisch alle Prüfer freundlich und beruhigend hinweg, schließlich soll Wissen geprüft werden, nicht Stressresistenz. Am besten schaut man dazu in eines der Prüfungsprotokolle, oft beruhigt das schon stark.

Die neuen Regeln bezüglich Abmeldung, Abwahl und nachträglichen Notenaufbesserung von Fächern machen es außerdem recht risikofrei, eine schwer wirkende Prüfung einfach mal auf sich zu kommen zu lassen. Auf keinen Fall sollte man sich vom scheinbar unerreichbaren Niveau einschüchtern lassen und den Großteil der Prüfungen last-minute abmelden.

\subsection{Selbstständiges Nachlernen von Bachelor-Fächern}
Unabhängig von Niveau und Anspruch hat dein Bachelor vielleicht eine andere Ausrichtung gehabt als man es hier gewohnt ist und somit in manchen Bereichen klare Wissenslücken hinterlassen. Wenn du das Gefühl hast, dass dir Wissen fehlt, das im Braunschweiger Bachelor vermittelt wird, kannst du dich natürlich auch freiwillig in jede Bachelor-Vorlesung oder Übung hineinsetzen - Punkte gibts dafür allerdings keine. Aber egal was dir aus dem Bachelor fehlt, es finden sich eigentlich stets genug Master-Fächer die auch ohne bestimmte Vorkenntnisse gut schaffbar sind. Einige wenige Master-Vorlesungen beginnen auch mit einer mehrwöchigen Widerholung der Bachelor-Grundlagen. Im Zweifelsfall frage Studenden aus den höheren Semestern oder den Prof selbt, welche Vorkenntnisse man wirklich braucht.

\subsection{Stunden- und Semesterplan}
Im Vergleich zum Bachelorstudium an der TU-BS bietet der Master verblüffende Freiheiten in der Fächerwahl. Auch wer die Redewendung bisher dumm fand, lernt spätestens hier was "`die Qual der Wahl"' bedeutet.

Zu Beginn jedes Semesters muss man sich selbständig entscheiden, welche Fächer man belegen möchte (und kann) und sich danach den Stundenplan zusammenstellen. Im Gegensatz zu den alteingessenen Bacheloranden aus Braunschweig steht man recht unvorbereitet vor einer ganz und gar nicht trivialen Aufgabe, die man am besten noch vor dem ersten Studientag erledigen sollte. Vielleicht hatte dein Bachelor kaum Wahlmöglichkeiten und das Zusammenstellen eines eigenen Stundenplans ist für dich ungewohnt. Und selbst wenn du diese Freiheiten schon im Bachelor hattest, so sind die Rahmenbedinungen und Feinheiten hier sicherlich ganz andere.

Und anders als im Bachelor, bei dem die meisten Vorlesungen erst in der zweiten Woche beginnen, starten viele Mastervorlesungen schon in der ersten Woche, schlimmstenfall schon Montag früh. Die erste Vorlesung zu verpassen ist nun auch nicht so schlimm, aber wenn mans vermeiden kann\ldots Es ist übrigens auch nicht ganz leicht, den Vorlesungsstart eines Faches herauszufinden, somit wirst du in der ersten (und evtl. auch zweiten) Woche oft vor verschlossenen Türen oder leeren Räumen stehen.

\subsection{Wie viele Credit Points?}
Standardmäßig ist vorgesehen, pro Semester 30 Credit Points zu erlagen - so hat man nach 4 Semestern den Master in der Tasche. Man ist dann aber auch zeitlich sehr ausgelastet, und für Urlaub, Familie und Nebenjob bleibt nicht unbedingt Zeit. Wenn man außerdem mit Zulassungsauflagen gesegtnet ist, sind dies bis zu 15 weitere Credit Points, die man irgendwie auf die ersten beiden Semester aufteilen muss, und so lohnt es sich durchaus frühzeitig darüber nachzudenken, wie viele Semester man wirklich studieren möchte und wie viele Credit Points man pro Semester ableisten möchte und kann. Du wirst hier und da noch Gerüchte hören, dass man mindestens 15 CP pro Semester schaffen muss. Das war früher mal so, wurde aber glücklicherweise nun abgeschafft, also lass dich von solchen Aussagen nicht allzusehr beeinflussen.

Dann steht ja dem entspannten Studium (fast) nur noch die Finanzierung im Wege. BAFöG-Höchstförderungsdauer, Langzeitstudiengebühren, sowie das Ende von Kindergeld, Kindesunterhalt und Famlienversicherung bei der Krankenkasse sind hier die relevanten Stichwörter, die viele Masterstudenten irgendwann ereilen. Hiwi-Jobs, Studienkredite und Stipendien verschaffen vielleicht Linderung.

Was auch immer du nun denkst, wie viele CP du im kommenden Semester belegen möchte, plane vielleicht ein paar Reserve-Punkte ein, also zusätzliche Fächer, die du belegst. Du kannst dann immernoch im laufenden Semester mache Vorlesungen "`kicken"' wenn es doch nicht so spannend ist wie zuerst gedacht bzw. du kommst noch auf die angepeilte Punktezahl, auch wenn du durch ein oder zwei Prüfungen fällst. Durchfallen ist weder eine Schande noch ein großes Problem, da die Prüfungsordnung dir erlaubt, bis zu drei Fächer, bei denen du durchgefallen bist, so abzuwählen als hättest du sie nie belegt. Dennoch sollte man es vielleicht mit den Reservefächern nicht übertreiben. Versuche einfach, den folgenden plumpen Witz auf diese Situation zu übertragen, und frag dich, ob das zielführend ist:

Kunde: Ich hätte gerne 20 Brötchen. 
Bäckerin: So viele? Davon wird ihnen doch die Hälfte trocken bevor sie die gegessen haben!
Kunde: Oh, das hab ich nicht bedacht. Dann nehm ich doch lieber 40 Stück.

\subsection{Nebenfach und Studienrichtung}
Laut Prüfungsordnung steht es dir frei, ob du im Master ein Nebenfach wählen möchtest oder "`reine"' Informatik studierst. Die Nebenfach-Enscheidung will gut überlegt sein, denn wenn man erstmal "`drin"' ist (d.h. zwei Prüfungen im Nebenfach bestanden hat) kommt man nicht mehr raus. Die Studienrichtung ist ebenso optional, aber im Gegensatz zum Nebenfach geht man damit keinerlei Verpflichtung ein. Am Ende des Studiums wird einfach geschaut, ob man 70 Credit Points aus einem artverwanden Bereich gemacht hat und bekommt dann auf Wunsch ein Sonderprädikat aufs Zeugnis. Beide Entscheidungen musst du nicht im ersten Semester treffen, sondern kannst dich auch später (aber am besten nicht zu spät) spezialisieren.

\subsection{Mentoren und Beratungsgespräche}
Laut Studienordnung bekommst du auch einen Mentor zugewiesen - das ist Professor aus der Informatik, der dich bei Entscheidungen zum Studium im persönlichen Gespräch beraten soll. Gerade wenn du weißt, dass du dich spezialisieren möchtest, oder wenn du zumindest mit dem Gedanken spielst, solltest du einen Mentor haben, der aus der jeweiligen Fachrichtung kommt. Wird dir zu Beginn ein völlig fachfremder Mentor zugewiesen, dann kannst du recht formlos darum bitten, diesen zu wechseln. Gespräche mit dem Mentor sind weder verpflichtend noch planmäßig vorgesehen, es liegt also an dir, dich um einen Termin zu kümmen, wenn du beraten werden möchtest. Manche Mentoren veranstalten auch im Dezember ein großen Treffen mit all ihren Schützlichen, was dich aber nicht davon abhalten sollte, schon vorher das Gespräch zu suchen.

Außer dem dir zugewiesenen Mentor gibt es noch weitere Ansprechpartner für verschiedenste Anlässe. Die wichtigsten haben wir für dich unter http://fginfo.cs.tu-bs.de/index.php/kontakt/ansprechpartner/ zusammengefasst.

\subsection{Modularten}
Den Hauptteil deiner 120 Credit Points sammelst du durch den Besuch von Vorlesungen und durch Bestehen der damit zusammenhängenden Prüfungen. Es ist schon schwer genug, sich für eine Menge von Vorlesungen zu entscheiden, aber hinzu kommen noch diverse andere Arten, sich Punkte zu verdienen. Das meiste davon gilt auch schon im Bachelor, siehe dazu den entsprechenden Abschnitt auf Seite ??. Für den Master kommt noch die Projektarbeit hinzu. Dies ist ein dicker, optionaler Brocken der euch 14 CP einbringt, und der in einem eigenständig erstellten Software-Projekt und schriftlicher Ausarbeitung besteht. Wenn man solch ein Projekt einbringt, dann überlicherweise direkt vor der Masterarbeit, es wird dich daher im ersten Semester noch nicht direkt interessieren, ist aber der vollständigkeit halber erwähnt.

\subsection{Mündliche Prüfungen}
Übrigens wirst du feststellen, dass fast alle Masterfächer mit einer mündlichen Prüfung abschließen. Wer keine Pflichtauflagen hat, kann eventuell durch das gesamte Masterstudium ohne eine einzige schriftliche Klausur kommen! Im Gegensatz zu Klausurabschriften ist es generell geduldet, Prüfungsprotokolle zu verteilen. Das sind Gedächtnisprotokolle dessen, wie eine Prüfung abgelaufen ist, also was für Fragen gestellt werden, wie der Prof auf falsche Antworten reagiert, etc. Du findest diese Protokolle auf den Seiten der Fachgruppe.

\subsection{Studienplan und Reihenfolge}
Mindestens so frei wie die Wahl der Fächer ist auch die Wahl der Reihenfolge, in der man diese belegt. Es kursieren Studienpläne für den Master, die vorschlagen, bestimmte Module im ersten Semester zu machen (z.B. das Seminar), und andere im zweiten und dritten, etc. All dies sind höchstens gut gemeinte Tipps, letzlich muss ja nichtmals die Masterarbeit unbedingt am Ende stehen, und zur Idee, das Seminar im ersten Semester zu belegen, siehe auch den nächsten Absatz\ldots

\subsection{Seminar}
Im Master musst du auch ein so genanntes Seminar einbringen, das ist eine Ausarbeitung zu einem Thema, die meist in einem Vortrag und einer mehrseitigen schriftlichen Ausarbeitung mündet. Anders als für alle anderen Modularten muss man sich für das Seminar inklusive Themenwahl schon im Vorraus anmelden. Die Frist für das jetzt beginnende Wintersemester war irgendwann Anfang August, daher wird wohl kein jetzt hinzugezogener Masterstudent sein Seminar im ersten Semester belegen. Keine Sorge, du hast noch mindestens drei weitere Semester, in denen du dieses Modul unterbringen kannst. Halte einfach kurz vor Vorlesungsende Ausschau nach der Ankündigung der Seminar-Info-Veranstaltung, z.B. auf der cs-studs Mailingliste.

\subsection{Welche Fächer gibt es?}
Die Liste der Fächer ist groß und ständig im Wandel. Offiziell festgelegt sind diese im Modulhandbuch (MHB), und anders als der Name vermuten lässt, präsentiert sich dieses nicht etwa als handliches Buch für die linke Jackeninnentasche, sondern als recht unübersichtliche Webanwendung. Unter \url{https://mhb.tu-bs.de/mhb1011ws/studiengangAbstract.do?id=176&call=veranstaltungAbstract.do%3Fid%3D1818} findest du eine Liste sämtlicher Module, die du im Master einbringen kannst. Und da so ein kryptischer Link weder angenehm zu tippen ist, noch garantiert ist, dass er am Tage nach dem Druck dieses Heftes noch erreichbar ist, kannst du auch \url{https://mhb.tu-bs.de/mhb1011ws/} aufrufen und dann über "`Studiengänge ansehen"' navigieren um dann nach "`Informatik Master"' Ausschau zu halten.

All diese Fächer kannst du als Masterstudent belegen - aber längst nicht jedes davon wird in diesem Semester angeboten. Nach einem Klick auf ein Fach siehst du die Details. Dort steht dann auch alles weitere zum Modul, und manches davon ist verbindlich und stets aktuell und korrekt. Die Information, ob ein Fach im Winter oder Sommer angeboten wird, gehört definitiv nicht dazu, was uns zur nächsten Informationsquelle bringt\ldots

\subsection{Der generelle Stundenplan}
Unter \url{http://www.cs.tu-bs.de/stundenplan/ws1011/wochenplan.do-20100723.html} findest du den aktuellen Plan für das Wintersemester. Dort sind - theoretisch - alle Veranstaltungen der Informatikmodule eingetragen, inklusive der Nebenfächer, jedoch ohne Schlüsselqualifikations-Pool (siehe entsprechender Abschnitt weiter unten). Der Stundenplan enthält sowohl Bachelor- als auch Masterfächer. Der Plan ist nicht getrennt, da nämlich die Bachelorstudenten auch ein paar Master-Fächer einbringen dürfen - andersrum gilt das aber nicht. Also musst du für jedes Fach, was du hier findest, erstmal verifizieren ob du dessen Punkte überhaupt im Master einbringen kannst. Wie du dir vielleicht schon denken kannst, wird dein persönlicher Stundenplan eine Untermenge dieses Mammut-Plans, erweitert um ein paar Veranstaltungen die selbst hier nicht stehen.

Wenn etwas darauf hindeutet, dass eine bestimmte Vorlesung im Semester angeboten wird, diese aber im Stundenplan nicht auftaucht, dann hilft eine Suche auf den Institusseiten, und wenn selbst das nicht hilft, eventuell eine Mail an den verantwortlichen Professor. Das gleiche gilt, wenn irgendwas komisch wirkt, z.B. wenn im Stundenplan zu einem Fach 5 Übungstermine und kein Vorlesungstermin stehen, was nun durchaus nicht das erste Mal wäre.

\subsection{Schlüsselqualifikationen}
Zu deinem Master-Abschluss gehören auch 8 bis 10 Punkte aus dem Schlüsselqualifikations-Pool. Das ganze heißt deshalb Pool, weil darin Vorlesungen von allen Fachbereichen und Instituten der Uni schwimmen, nicht nur Informatikbezogene. Da wir hier von 100 bis 300 angebotenen Fächern je Semester reden, sind diese nicht im Modulhandbuch und im Informatik-Studenplan vermerkt, sondern können irgendwo auf \url{http://www.tu-braunschweig.de/studium/lehrveranstaltungen/fb-uebergreifend} gefunden werden.

\subsection{Der eigene Stundenplan}
Es gibt irgendwo in den überabzählbar-unendlichen Weiten der TU-BS-Webseiten auch ein Tool names QIS oder QIP oder HIS oder sonstwie, mit dem du dir die eben erwähnte Untermenge zusammenstellen, speichern und ausdrucken kannst. Dort sind (bzw. waren vor einem Jahr) aber viele Fächer nicht eingetragen, was das Tool dann eher nutzlos macht. Parallel dazu gibt es noch das Stud.IP-Portal, welches ähnliche Funktionen anbietet, aber vermutlich noch unvollständiger und somit nutzloser ist. Wahrscheinlich hilft also nichts außer ein selbst erstellter Stundenplan. Wie kommt man also dahin?

Es gibt durchaus Studenten, die damit kein Problem haben: Sie schauen einige Minuten auf den Gesamstundenplan, es macht Klick, und sie wissen, welche Fächer sie belegen werden. Es gibt andere, nicht weniger schlaue, die bis zu 12 Stunden damit verbringen, bis sie ihren finalen Stundenplan beieinander haben. Falls du nicht zum unteren Extrem gehörst, soll dir dieser Text helfen, auch nicht zum oberen zu gehören:

Wenn du Zulassungsauflagen hast, haben diese oberste Priorität. Die entsprechenden Vorlesungen und Übungen kannst du ohne großes Nachdenken in deinen Stundenplan eintragen - außer wenn du die freiwillige mündliche Prüfung in Anspruch genommen und bestanden hast, bzw. du guter Hoffnung bist, sie zu bestehen.

Danach kannst du probieren, im allgemeien Stundenplan pro Block durchzugehen, und für jeden Block zu entscheiden, welches der dort stattfindenen Fächer für dich interessant klingt, und dieses herausschreiben oder markieren. Wenn du so vorgehst, hast du vermutlich am Ende einen Plan mit viel zu vielen Fächern, also deutlich mehr als 30 Credit Points. Und was zu Beginn noch Überscheidungsfrei aussieht, endet am Ende vielleicht in folgender Situation:

Vorlesung A überschneidet sich mit Vorlesung B und einem Übungstermin von Fach C. Der andere Übungstermin von Fach C kollidiert mit Vorlesung D, deren einzige Übung mit Übung A am Dienstag zusammenstößt, die man alternativ auch am Mittwoch haben könnte, was sich dann aber so halb mit der Vorlesung F überschneidet\ldots

Vielleicht springt dir nun sofort eine Lösung ins Auge. Falls nicht, hier noch ein paar Fälle, in denen eine vermeintliche Kollision gar keine ist, oder zumindest kein wirkliches Problem darstellt:

Manche Übungen finden nur alle zwei Wochen statt. Wenn also in einem Block die Übung zu Fach A und zu Fach B liegen, dann könntest du Glück haben, dass sich diese genau abwechseln. Dann ist aber wieder Vorsicht geboten, da die Lehrenden oft (z.B. wegen Urlaub, Krankheit, Konferenzen, Feiertag\ldots irgendein Grund findet sich immer) die Regelmäßigkeit mitten im Semester brechen und die zuvor abwechselnden Übungen dann wieder aufeinander liegen.

Man muss nicht immer beide Veranstaltungen besuchen: bei manchen Fächern kann man die Übung getrost weglassen, oder den Stoff auch ohne Vorlesung aus Skript und Büchern lernen und nur zur Übung kommen. Oder wenn sich Vorlesung X und die 14-täglich stattfindende Übung Y überschneiden, so kommt man halt nur alle zwei Wochen zur Vorlesung X. Nicht toll, nicht angenehm, aber oft machbar. Manche Institute filmen ihre Vorlesungen auch und machen sie somit auch zeitversetzt studierbar. Frage am besten höhere Semester nach ihren Erfahrungen mit dem betreffenden Fach.

Wenn es für ein Fach mehrere Übungstermine gibt, so sind diese meist für mehrere Übungsgruppen vorgesehen - du bist dann nur in einer dieser Gruppen und besuchst nur einen dieser Termine. Die Gruppe kann man meist frei wählen.

Außerdem passiert es recht oft, dass in den ersten Wochen noch Übungstermine bedarfsgerecht verschoben werden. Dadurch könenn sich Überschneidungen auflösen - aber natürlich auch neue dazu kommen.

\subsection{Immernoch keine Lösung?}
Nun ist Kreativität gefragt: Wende an, was du im Bachelor gelernt hast. Stelle die Kollisionen als Graph oder Matrix oder Tupelmenge dar, und lasse ein paar Algorithmen darauf los, die du dir ausdenkst und dann auf dem Papier simulierst. Bastel eine Excel-Tabelle mit Formeln und Makros oder schreibe ein kleines Programm, dass den optimalen Stundenplan berechnet. 

Spaß beiseite, auch die Auflistung in ihrer Schreibweise und Anhäufing leicht ironisch wirkt, sind das durchaus ernst gemeinte Vorschläge. In Extremfällen kann die Sache so vertrackt werden, dass ein paar hundert oder gar tausend Alternativen zu vergleichen sind. Von Hand kann und will das keiner, aber wenn zwei bis drei Stunden Informatiker-typisches gefrickel am Rechner dazu führen, den perfekten Stundenplan fürs nächste Semester zu finden, dann ist es die Sache doch wert.

\subsection{Hilfe beim Stundenplanbau}
Wie bieten dieses Jahr auch erstmals Hilfe zum Stundenplanbau an. Hätten wir das schon zu Beginn des Textes erwähnt, dann hätte ja niemand weitergelesen ;) Im Ernst: Würden zu Semesterbeginn 50 Stundenten unvorbereitet mit leeren Pänen zu uns kommen, so könnten wir jedem nur wenige Minuten helfen und letzlich wäre keinem so richtig geholfen. Wir hoffen, dass die obigen Ausführungen den meisten von euch reichen, um auch allein zum fast perfekten Stundenplan zu kommen - wenn es denn sowas gibt. Wer trotzdem noch total auf dem Schlauch steht, kann nun gerne zu uns kommen und sich intensiv und persönlich beraten lassen. Und auch wer glaubt, jetzt alles verstanden zu haben kann mit seinem fertigen Stundenplan gerne zu uns kommen, und wir schauen kurz drauf ob uns offensichtliche Probleme oder Verbesserungsvorschläge auffallen. Wir bitten dich also, vor dem Planbau-Termin ernsthaft zu versuchen, selbst zu einem Plan zu kommen. Und nicht nach 5 Minuten aufgeben: zwei bis drei Stunden kannst du ruhig dafür einplanen. Zu den oben genannten 12 Stunden muss man es ja nicht kommen lassen.

\subsection{Und nun?}
Das war nun eine ganze Menge Text. Und nun weißt du alles, was du für deinen Weg zum Master-Abschluss brauchst? Ganz bestimmt nicht. Dies war die Notfallration für die ersten paar Tage und Wochen. In der ersten Semesterwoche gibt es eine ganze Reihe von Infoveranstaltungen, einige Wochen oder Monate später auch noch vereinzelze Termine mit den Infos, die du dann gerade brauchen wirst, und ab da hoffen wir, dass du dich gut eingefunden hast und dass bis dahin dein Jahrgang soweit zusammengewachsen ist, dass ihr euch gegenseitig auf dem Laufenden haltet bzw. Kontakt zu den höheren Semestern habt. Wenn doch noch Fragen bestehen, so gibts immer noch uns (die Fachgruppe) und diverse andere Ansprechpartner. Aber eine so geballte Packung Infos wirst du von hier an wohl nie wieder im Studium brauchen.


%Nicht nur für Master relevant (vieles davon steht vielleicht schon drin, hab die Liste geschrieben als ich die nte gerade nicht zur Hand hatte):
%Drucken
%GIZT-Konto
%ÖPNV
%Mensa
%Lageplan und Räume
%TUBS-Webseite und Institutsseiten
%Mailinglisten
%Bücher
%Lernräume
%Vollversammlungen
%Studiengebühren
%Wahlen
%Evaluationsbögen
%Prüfungsprototkolle
