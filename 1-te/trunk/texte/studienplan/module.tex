\subsection{Module und Co.}
Um euren Abschluss zu bekommen - und deshlab seid ihr ja hoffentlich hier - müsst ihr eine vordefinierte Menge von Bereichen abdecken. Die Bereiche unterscheiden sich inhaltlich und formal, aber allesamt sind so genannte \textit{Module}. Da dies so zentral für euer Studium sind, möchten wir diese im Detail erkären.

\end{multicols}

\subsubsection{Arten von Modulen}

Die Folgende Tabelle sagt dir, welche Modularten es gibt und wiele Creditpoints ein solches Modul gibt, bzw. wie viele CP dieser Art du im Studium einbringen darfst.

\begin{tabular}{|p{3mm}|c|c|c|c|c|}
\hline & \textbf{Modulart}		& \textbf{CP je Modul} 	&\textbf{im Bachelor}	 	& \textbf{im Master}		& \textbf{Benotet?} \\ 
\hline & Vorlesung mit Übung	& 5 bis 10 				& viele						& viele						& ja \\ 
\hline & Seminar				& 5 	 				& 5 (evtl. mehr?)			& 5 (evtl. mehr?)			& ja \\ 
\hline & Schlüsselquali			& 1 bis 8 				& 8							& 8							& nein \\ 
\hline & SEP					& 8		 				& 8							& 0							& nein \\ 
\hline & Praktikum				& 5 bis 10 				& viele						& viele						& nein \\ 
\hline & Teamprojekt			& 5 bis 10 				& viele						& 0							& nein \\ 
\hline & Projektarbeit			& 14	 				& 0							& 14						& ja \\ 
\hline & Bachelorarbeit			& 15					& 15						& 0							& ja \\ 
\hline & Masterarbeit			& 30	 				& 0							& 30						& ja \\ 
\hline
\end{tabular} 
\begin{multicols}{2}

Den Hauptteil deiner 120 Credit Points sammelst du durch den Besuch von Vorlesungen und durch Bestehen der damit zusammenhängenden Prüfungen. Es ist schon schwer genug, sich für eine Menge von Vorlesungen zu entscheiden, aber hinzu kommen noch diverse andere Arten, sich Punkte zu verdienen. Das meiste davon gilt auch schon im Bachelor, siehe dazu den entsprechenden Abschnitt auf Seite ??. Für den Master kommt noch die Projektarbeit hinzu. Dies ist ein dicker, optionaler Brocken der euch 14 CP einbringt, und der in einem eigenständig erstellten Software-Projekt und schriftlicher Ausarbeitung besteht. Wenn man solch ein Projekt einbringt, dann überlicherweise direkt vor der Masterarbeit, es wird dich daher im ersten Semester noch nicht direkt interessieren, ist aber der vollständigkeit halber erwähnt.

\subsubsection{Vorlesung, Übung, etc.}
\paragraph*{Vorlesung}

Vorlesungen werden vom Professor vor allen Studis abgehalten und befassen
sich in erster Linie mit der theoretischen Herleitung des Stoffes. Teilweise
sind Vorlesungen aber auch nur mehr oder weniger interessante Folienfilme auf
dem Overhead-Projektor. Solltest du in der Vorlesung einmal etwas nicht
verstehen, so ist das nicht so tragisch, den meisten deiner Kommilitonen geht
es nicht anders. Schau dich mal um und du wirst viele andere fragende Gesichter
sehen\ldots Du darfst nicht damit rechnen, wie in der Schule, das meiste sofort zu
verstehen, f"ur jede Vorlesung sollte man eine gewisse Nacharbeitungszeit
einplanen. In einer Vorlesung ist wegen der gro"sen Teilnehmerzahl
normalerweise kein Dialog mit dem Vortragenden m"oglich. Aufgetretene Fragen
k"onnen und sollten am besten direkt nach der Vorlesung oder sonst in einer
Sprechstunde mit dem Professor gekl"art werden.


\paragraph*{Gro"se "Ubung}

Erg"anzend gibt es die gro"sen "Ubungen, auch Saal"ubungen genannt. Diese
finden~-- wie die Vorlesung~-- vor dem gesamten Auditorium statt und sollen das
(vielleicht) erworbene theoretische Wissen vertiefen und vor allem auch
praktische, klausurbezogene Anwendungen aufzeigen. Die gro"se "Ubung wird
normalerweise von einem Assistenten gehalten, selten vom Professor selbst.
Assistenten ("`Assis"') sind fertige Dipl.-Ings, Dipl.Informs etc. und sind
Angestellte des Instituts, aus dem auch der jeweilige Professor stammt. Die
Assis sind bei fachlichen Fragen kompetente Ansprechpartner und meist auch sehr
hilfsbereit. Da Assistenten "ublicherweise die Klausuren entwerfen, kann man
bei genauem Hinh"oren in den gro"sen "Ubungen oder im privaten Gespr"ach mit
dem Assi einiges "uber den Tag der Wahrheit erfahren.


\paragraph*{Kleine "Ubung, Seminargruppe}

Als erstes eine Warnung: Kleine "Ubungen tauchen in deinem Stundenplan nicht auf!
Also f"ull bitte nicht alle L"ucken im
Stundenplan mit Sprachkursen, Sportveranstaltungen und Klavierunterricht auf,
sondern lass noch ein bisschen Platz. Leider werden kleine "Ubungen nur in
einigen F"achern angeboten. Der Begriff Seminargruppe ist synonym zu verstehen.
In kleinen "Ubungen soll man eigentlich selbst Aufgaben l"osen. Dies geschieht
unter Anleitung der HiWis (Hilfswissenschaftler), welche besonders qualifizierte
(!?) Studierende h"oheren Semesters sind. F"ur die kleinen "Ubungen werden die
Studis in etwa 20- bis 30-k"opfige Gruppen aufgeteilt. Hierbei ist darauf zu
achten, rechtzeitig zum Termin zur Gruppeneinteilung zu erscheinen, um diese
Veranstaltungen m"oglichst g"unstig im Stundenplan positionieren zu k"onnen.
Manche Assistenten haben inzwischen auch Methoden entwickelt, bei denen man
ohne Ellenbogen einen passenden Termin bekommt, aber das hat sich noch nicht
vollst"andig durchgesetzt. Aufgrund der geringen Teilnehmerzahlen ist in
kleinen "Ubungen der Dialog mit dem Vortragenden m"oglich und sinnvoll. Wenn
man einen guten HiWi erwischt hat, dann kann man in den kleinen "Ubungen all
die Wissensl"ucken auff"ullen, die nach Vorlesung und gro"ser "Ubung noch offen
sind.

%TODO schreiben
\paragraph*{Klausur}
kein Text

%TODO zu masterspezifisch
\paragraph*{Mündliche Prüfungen}
Übrigens wirst du feststellen, dass fast alle Masterfächer mit einer
 mündlichen Prüfung abschließen. 
 Wer keine Pflichtauflagen hat, kann eventuell durch das gesamte
 Masterstudium ohne eine einzige schriftliche Klausur kommen! 
Im Bachelor sind hingen nahezu alle Prüfungen schrifltich, laut
 Prüfungsordnung sind aber drei mündliche Prüfungen abzulegen.
Im Gegensatz zu Klausurabschriften ist es generell geduldet, Prüfungsprotokolle zu verteilen. Das sind Gedächtnisprotokolle dessen, wie eine Prüfung abgelaufen ist, also was für Fragen gestellt werden, wie der Prof auf falsche Antworten reagiert, etc. Du findest diese Protokolle auf den Seiten der Fachgruppe.

\subsubsection{Seminar}
Außerdem musst du auch sowohl im Bachelor als auch im Master ein so genanntes Seminar einbringen, das ist eine Ausarbeitung zu einem Thema, die meist in einem Vortrag und einer mehrseitigen schriftlichen Ausarbeitung mündet. Anders als für alle anderen Modularten muss man sich für das Seminar inklusive Themenwahl schon im Vorraus anmelden. Halte einfach kurz vor Vorlesungsende Ausschau nach der Ankündigung der Seminar-Info-Veranstaltung, z.B. auf der cs-studs Mailingliste. Die Frist für das jetzt beginnende Wintersemester war irgendwann Anfang August, daher wird wohl kein jetzt hinzugezogener Masterstudent sein Seminar im ersten Semester belegen.

Prinzipiell kannst du dir, wie bei den meisten Modulen, aussuchen, in welchem Semester du das Seminar einbringst. Leider denken viele, man müsste das Seminar im 5. Bachelorsemester oder im 1. Master-Semester einbringen. Daher sind die Seminare im Wintersemester oft überbucht, und im Sommersemester sind noch viele Themen frei. Wenn du also ein Thema abbekommen möchtest, dass dir auch wirklich gefällt, solltest du darüber nachdenken, es in ein Sommersemester zu verlegen. Außerdem kannst du dich in einem recht frühen Semester (also z.B. im 3. oder 4.) auf ein Seminar bewerben, und dann, wenn du kein wirklich interessantes Thema abbekommst, im nächsten Semester versuchen, etwas besseres zu bekommen.

%TODO Dieser Absatz ist unverständlich, und danch kommt nochmal einer mit Dopplungen. Überarbeiten!
\subsubsection{Schlüsselqualifikationen/Mathe-Wahlpflicht}
Zu deinem Master-Abschluss gehören auch 8 bis 10 Punkte aus dem
Mathe-Wahlpflicht Schlüsselqualifikations-Pool. Im Bachelor sind dort
10 Credits einzubringen. Das ganze heißt deshalb Pool, weil darin
Vorlesungen von allen Fachbereichen und Instituten der Uni schwimmen,
nicht nur Informatikbezogene. Da wir hier von 100 bis 300 angebotenen
Fächern je Semester reden, sind diese nicht im Modulhandbuch und im
Informatik-Studenplan vermerkt, sondern können irgendwo auf
\url{http://www.tu-braunschweig.de/studium/lehrveranstaltungen/fb-uebergreifend}
gefunden werden. Außerdem können hier noch Wahlpflichtfächer aus der
Mathematik eingebracht werden.  Zu beachten ist dabei, dass man dabei
nur Fächer belegen darf, die nicht aus den Nebenfach kommen. Man kann
also z.B mit den Nebenfach Mathe nicht Mathewahlpflichtfächer
belegen. Diese Regelungen sind im Bachelor im Wesentlichen die
Gleichen. Allerdings ist dort der Mathe-Wahlpflichtbereich ein eigener
Bereich von 10 Credits und gehört nicht zu den Schlüsselqualifikationen.

\paragraph*{Nochmal Schl"usselqualifikationen}
Jeder Bachelorstudent muss sogenannte Schl"usselqualifikationen innerhalb seines Studiums erwerben.
In der Informatik m"ussen dies "`handlungsorientierte Anwendungen"' im Umfang von 10 Leistungspunkten sein.
Hierzu z"ahlen Sprachkurse und "uberfachliche Lehrveranstaltungen.
Informationen "uber das aktuelle Angebot (Pool-Modell) und die zu erf"ullenden Bestimmungen der Veranstaltungen findet ihr auf dieser Webseite: \nurl{http://www.tu-braunschweig.de/informatik-bsc/struktur/schluessel}

\paragraph*{Sprachenzentrum}
Am Sprachenzentrum der Uni kannst du verschiedene Sprachkurse belegen, die du auch als Schl"usselqualifikationen in deinen Bachelor-Abschluss einbringen kannst.
Auf den Seiten des Sprachenzentrums (\nurl{www.sz.tu-bs.de}) findest du alle angebotenen Kurse.
Um sich f"ur Kurse anzumelden, brauchst du ein Konto, das du pers"onlich in der Mediothek (im Altgeb"aude \nurl{http://www.sz.tu-bs.de/mediothek/}) registrieren musst.

\textbf{Wichtig!} Die Anmeldung f"ur Sprachkurse beginnt bereits vor jedem Semester.
Um Pl"atze zu bekommen, solltest du dich also so fr"uh wie m"oglich anmelden.
Bevor du an einem Englischkurs teilnehmen kannst, musst du zun"achst einen Einstufungstest machen.
Die Termine findest du hier: \nurl{http://www.sz.tu-bs.de/fremdsprachen/englisch/einstufungstest/}\\
Da gerade bei diesen Kursen die Nachfrage sehr hoch ist, solltest du den Test m"oglichst bereits vor dem Anmeldungszeitraum (beginnt etwa 2 Wochen vor Vorlesungsbeginn) ablegen.

\paragraph*{Praktikum}
Ja, hier muss noch ein bisschen Text rein.

\paragraph*{SEP (Software-Entwicklungs-Praktikum)}
Ja, hier muss noch ein bisschen Text rein.

\paragraph*{Teamprojekt}
Ja, hier muss noch ein bisschen Text rein.

\paragraph*{Projektarbeit im Master}
Ja, hier muss noch ein bisschen Text rein.

\paragraph*{PADI (Praktische Aspekte der Informatik)}
Ja, hier muss noch ein bisschen Text rein.

\subsubsection{Abschlussarbeit}
Ja, hier muss noch ein bisschen Text rein.